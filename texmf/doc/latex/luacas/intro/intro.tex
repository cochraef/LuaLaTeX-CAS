\documentclass{article}

\usepackage{luacas}
\usepackage{amsmath}
\usepackage{amssymb}

\usepackage[margin=1in]{geometry}
\usepackage[shortlabels]{enumitem}

\usepackage{pgfplots}
\pgfplotsset{compat=1.18}
\usetikzlibrary{positioning,calc}
\usepackage{forest}
\usepackage{minted}
\usemintedstyle{pastie}
\usepackage[hidelinks]{hyperref}
\usepackage{parskip}
\usepackage{multicol}
\usepackage[most]{tcolorbox}
    \tcbuselibrary{xparse}
\usepackage{microtype}

\usepackage[
backend=biber,
style=numeric,
]{biblatex}
\addbibresource{sources.bib}

\newtcolorbox{codebox}[1][sidebyside]{
    enhanced,skin=bicolor,
    #1,
    arc=1pt,
    colframe=brown,
    colback=brown!15,colbacklower=white,
    boxrule=1pt,
    notitle
}

\begin{document}

\section{What is \texttt{luacas}?}

The package {\ttfamily luacas} allows for symbolic computation within \LaTeX{}. For example:
\begin{CAS}
    vars('x','y')
    f = 3*x*y - x^2*y
    fxy = diff(f,x,y)
\end{CAS}
\begin{minted}[fontsize=\small]{latex}
\begin{CAS}
    vars('x','y')
    f = 3*x*y - x^2*y
    fxy = diff(f,x,y)
\end{CAS}
\end{minted}
The above code will compute the mixed partial derivative $f_{xy}$ of the function $f$ defined by
\[ f(x,y)=3xy-x^2y.\]
There are various methods for fetching and/or printing results from the CAS within your \LaTeX{} document:

\begin{codebox}
\begin{minted}{latex}
    \[ \print{fxy} = \print*{fxy} \]
\end{minted}
\tcblower
\[ \print{fxy} = \print*{fxy} \]
\end{codebox}

\subsection{About}

The core CAS program is written purely in Lua and integrated into \LaTeX{} via Lua\LaTeX{}. Currently, most existing computer algebra systems such as Maple and Mathematica allow for converting their stored expressions to \LaTeX{} code, but this still requires exporting code from \LaTeX{} to another program and importing it back, which can be tedious.

The target audience for this package are mathematics students, instructors, and professionals who would like some ability to perform basic symbolic computations within \LaTeX{} without the need for laborious and technical setup. But truly, this package was born out of a desire from the authors to learn more about symbolic computation. What you're looking at here is the proverbial ``carrot at the end of the stick'' to keep our learning moving forward.

Using a scripting language (like Lua) as opposed to a compiled language for the core CAS reduces performance dramatically, but the following considerations make it a good option for our intentions:

\begin{itemize}
    \item Compiled languages that can communicate with \LaTeX{} in some way (such as C through Lua) require compiling the code on each machine before running, reducing portability.
    \item Our target usage would generally not involve computations that take longer than a second, such as factoring large primes or polynomials.
    \item Lua is a fast scripting language, especially when compared to Python, and is designed to be compact and portable.
    \item If C code could be used, we could tie into one of many open-source C symbolic calculators, but the point of this project was (and continues to be) to learn the mathematics of symbolic computation. The barebones but friendly nature of Lua made it an ideal language for those intents.
\end{itemize}

\subsection{Features}

Currently, {\ttfamily luacas} includes the following functionality:

\begin{itemize}
    \item Arbitrary-precision integer and rational arithmetic
    \item Number-theoretic algorithms for factoring integers and determining primality
    \item Constructors for arbitrary polynomial rings and integer mod rings, and arithmetic algorithms for both
    \item Factoring univariate polynomials over the rationals and over finite fields
    \item Polynomial decomposition and some multivariate functionality, such as pseudodivision
    \item Basic symbolic root finding and equation solving
    \item Symbolic expression manipulations such as expansion, substitution, and simplification
    \item Symbolic differentiation and integration
\end{itemize}

The CAS is written using object-oriented Lua, so it is modular and would be easy to extend its functionality (which we hope to do in the future).

\subsection{Acknowledgements}

We'd like to thank the faculty of the Department of Mathematics at Rose-Hulman Institute of Technology for offering constructive feedback as we worked on this project. A special thanks goes to Dr. Joseph Eichholz for his invaluable input and helpful suggestions.

\section{Installation}

\subsection{Requirements}

The \texttt{luacas} package (naturally) requires you to compile with Lua\LaTeX{} (Lua 5.3 or higher is required in particular). Beyond that, the following packages are needed:
\begin{multicols}{2}
{\ttfamily
\begin{itemize}
    \item xparse
    \item pgfkeys
    \item verbatim
    \item mathtools
    \item luacode
    \item iftex
    \item tikz/forest
    \item xcolor
\end{itemize}}
\end{multicols}
The packages {\ttfamily tikz}, {\ttfamily forest}, and {\ttfamily xcolor} aren't strictly required, but they are needed for drawing expression trees.

\subsection{Installing {\ttfamily luacas}}
The package manager for your local TeX distribution ought to install the package fine on its own. But for those who like to take matters into their own hands: unpack \texttt{luacas.zip} in the current working directory (or in a directory visible to TeX, like your local texmf directory), and in the preamble of your document, put:
\begin{minted}{latex}
\usepackage{luacas}
\end{minted}
That's it, you're ready to go.

\subsection{Todo}

Beyond squashing bugs that inevitably exist in any new piece of software, future enhancements to \texttt{luacas} may include:
\begin{itemize}
    \item Improvements to existing functionality, e.g., a more powerful \texttt{simplify()} command and more powerful expression manipulation tools in general, particularly in relation to complex numbers, a designated class for multivariable polynomial rings, irreducible factorization over multivariable polynomial rings, and performance improvements;
    \item New features in the existing packages, such as sum and product expressions \& symbolic evaluation of both, and symbolic differential equation solving;
    \item New packages, such as for logic (boolean expressions), set theory (sets), and linear algebra (vectors and matrices), and autosimplification rules and algorithms for all of them;
    \item Numeric functionality, such as numeric root-finding, linear algebra, integration, and differentiation;
    \item A parser capable of evaluating arbitrary LaTeX code and turning it into CAS expressions.
\end{itemize}


\end{document}