\documentclass{article}

\usepackage{luacas}
\usepackage{amsmath}
\usepackage{amssymb}

\usepackage[margin=1in]{geometry}
\usepackage[shortlabels]{enumitem}

\usepackage{pgfplots}
\pgfplotsset{compat=1.18}
\usetikzlibrary{positioning,calc}
\usepackage{forest}
\usepackage{minted}
\usemintedstyle{pastie}
\usepackage[hidelinks]{hyperref}
\usepackage{parskip}
\usepackage{multicol}
\usepackage[most]{tcolorbox}
    \tcbuselibrary{xparse,documentation}
\usepackage{microtype}
\usepackage{makeidx}
\usepackage{fontawesome5}

\usepackage[
backend=biber,
style=numeric,
]{biblatex}
\addbibresource{sources.bib}

\definecolor{rose}{RGB}{128,0,0}
\definecolor{roseyellow}{RGB}{222,205,99}
\definecolor{roseblue}{RGB}{167,188,214}
\definecolor{rosenavy}{RGB}{79,117,139}
\definecolor{roseorange}{RGB}{232,119,34}
\definecolor{rosegreen}{RGB}{61,68,30}
\definecolor{rosewhite}{RGB}{223,209,167}
\definecolor{rosebrown}{RGB}{108,87,27}
\definecolor{rosegray}{RGB}{84,88,90}

\definecolor{codegreen}{HTML}{49BE25}

\newtcolorbox{codebox}[1][sidebyside]{
    enhanced,skin=bicolor,
    #1,
    arc=1pt,
    colframe=brown,
    colback=brown!15,colbacklower=white,
    boxrule=1pt,
    notitle
}

\newtcolorbox{codehead}[1][]{
    enhanced,
    frame hidden,
    colback=rosegray!15,
    boxrule=0mm,
    leftrule=5mm,
    rightrule=5mm,
    boxsep=0mm,
    arc=0mm,
    outer arc=0mm,
    left=3mm,
    right=3mm,
    top=1mm,
    bottom=1mm,
    toptitle=1mm,
    bottomtitle=1mm,
    oversize,
    #1
}

\DeclareTotalTCBox{\lilcoderef}{O{} m m}{
    enhanced,
    frame hidden,
    colback=rosegray!15,
    enhanced,
    nobeforeafter,
    tcbox raise base,
    boxrule=0mm,
    leftrule=5mm,
    rightrule=5mm,
    boxsep=0mm,
    arc=0mm,
    outer arc=0mm,
    left=1mm,
    right=1mm,
    top=1mm,
    bottom=1mm,
    oversize,
    #1
}{\mintinline{lua}{#2} \mintinline{lua}{#3}}

\usepackage{varwidth}

\newtcolorbox{newcodehead}[2][]{
    enhanced,
    frame hidden,
    colback=rosegray!15,
    boxrule=0mm,
    leftrule=5mm,
    rightrule=5mm,
    boxsep=0mm,
    arc=0mm,
    outer arc=0mm,
    left=3mm,
    right=3mm,
    top=1mm,
    bottom=1mm,
    toptitle=1mm,
    bottomtitle=1mm,
    oversize,
    #1,
    fonttitle=\bfseries\ttfamily\footnotesize,
    coltitle=rosegray,
    attach boxed title to top text right,
    boxed title style={frame hidden,size=small,bottom=-1mm,
    interior style={fill=none,
    top color=white,
    bottom color=white}},
    title={#2}
}

\makeindex

\newcommand{\coderef}[2]{%
\begin{codehead}[sidebyside,segmentation hidden]%
    \mintinline{lua}{#1}%
    \tcblower%
    \begin{flushright}%
    \mintinline{lua}{#2}%
    \end{flushright}%
\end{codehead}%
}

\newcommand{\newcoderef}[3]{%
\begin{newcodehead}[sidebyside,segmentation hidden]{#3}%
    \mintinline{lua}{#1}%
    \tcblower%
    \begin{flushright}%
    \mintinline{lua}{#2}%
    \end{flushright}%
\end{newcodehead}%
}
\usepackage{marginnote}

\begin{document}
\setdescription{style=multiline,
        topsep=10pt,
        leftmargin=6.5cm,
        }

\subsection{Calculus Methods}

\newcoderef{function IntegralExpression.table(integrand, symbol)}{return Expression|nil}{integrand Expression, symbol SymbolExpression}

Attempts to integrate the \texttt{integrand} with respect to \texttt{symbol} by checking a table of basic integrals; returns nil if the integrand isn't in the table. For example:

\begin{codebox}
    \begin{minted}[fontsize=\small]{latex}
\begin{CAS}
  vars('x')
  f = IntegralExpression.table(sin(x),x)
  g = IntegralExpression.table(x*sin(x),x)
\end{CAS}
\[ f = \print{f} \qquad g = \print{g} \] 
\end{minted}
\tcblower
\begin{CAS}
    vars('x')
    f = IntegralExpression.table(sin(x),x)
    g = IntegralExpression.table(x*sin(x),x)
\end{CAS}
\[ f = \print{f} \qquad g = \print{g} \] 
\end{codebox}

\newcoderef{function IntegralExpression.linearproperties(integrand, symbol)}{return Expression|nil}{integrand Expression, symbol SymbolExpression}

Attempts to integrate the \texttt{integrand} with respect to \texttt{symbol} by using linearity properties (e.g. the integral of a sum/difference is the sum/difference of integrals); returns nil if any individual component cannot be integrated using \mintinline{lua}{IntegralExpression.integrate()}. For example:

\begin{codebox}[]
    \begin{minted}[fontsize=\small]{latex}
\begin{CAS}
    vars('x')
    f = IntegralExpression.linearproperties(sin(x) + e^x,x)
    g = IntegralExpression.table(sin(x)+e^x,x)
\end{CAS}
\[ f = \print*{f} \qquad g = \print*{g} \]
\end{minted}
\tcblower
\begin{CAS}
    vars('x')
    f = IntegralExpression.linearproperties(sin(x) + e^x,x)
    g = IntegralExpression.table(sin(x)+e^x,x)
\end{CAS}
\[ f = \print*{f} \qquad g = \print*{g} \] 
\end{codebox} 

\newcoderef{function IntegralExpression.substitutionmethod(integrand, symbol)}{return Expression|nil}{integrand Expression, symbol SymbolExpression} 

Attempts to integrate \texttt{integrand} with respect to \texttt{symbol} via $u$-substitution; returns nil if no suitable substitution is found to be successful. 

\begin{codebox}[]
    \begin{minted}[fontsize=\small]{latex}
\begin{CAS}
    f = IntegralExpression.substitutionmethod(x*sin(x^2),x)
    g = IntegralExpression.linearproperties(x*sin(x),x)
\end{CAS}
\[ f = \print*{f} \qquad g = \print*{g}.\] 
\end{minted}
\tcblower
\begin{CAS}
    f = IntegralExpression.substitutionmethod(x*sin(x^2),x)
    g = IntegralExpression.linearproperties(x*sin(x),x)
\end{CAS}
\[ f = \print*{f} \qquad g = \print*{g}.\] 
\end{codebox}

\newcoderef{function IntegralExpression.trialsubstitutions(integrand)}{return table<number, Expression}{integrand Expression}

Generates a list of possible $u$-substitutions to attempt in \mintinline{lua}{IntegralExpression.substitutionmethod()}. 

\begin{codebox}[]
    \begin{minted}[fontsize=\small]{latex}
\begin{CAS}
    vars('x')
    f = cos(x)/(1+sin(x))
    f = f:autosimplify()
    l = IntegralExpression.trialsubstitutions(f)
\end{CAS}
\[ \left\{ \lprint{l} \right\}.\] 
\end{minted}
\tcblower
\begin{CAS}
    vars('x')
    f = cos(x)/(1+sin(x))
    f = f:autosimplify()
    l = IntegralExpression.trialsubstitutions(f)
\end{CAS}
\[ \left\{ \lprint{l} \right\}.\] 
\end{codebox}

\newcoderef{function IntegralExpression.rationalfunction(integrand, symbol)}{return Expression|nil}{integrand Expression, symbol SymbolExpression}

Integrates \texttt{integrand} with respect to \texttt{symbol} via Lazard, Rioboo, Rothstein, and Trager's method in the case when \texttt{expression} is a rational function in the variable \texttt{symbol}. If \texttt{integrand} is not a rational function, then nil is returned. 

\newcoderef{function IntegralExpression.partsmethod(integrand, symbol)}{return Expression|nil}{integrand Expression, symbol SymbolExpression} 

Attempts to integrate \texttt{integrand} with respect to \texttt{symbol} via \emph{integration by parts}; returns nil if no suitable application of IBP is found. 

\newcoderef{function IntegralExpression.eulersformula(integrand, symbol)}{return Expression|nil}{integrand Expression, symbol SymbolExpression}

Attempts to integrate \texttt{integrand} with respect to \texttt{symbol} by using Euler's formula ($e^{a+bi} = e^a (\cos(b) + i\sin(b))$) and kind. 

\newcoderef{function IntegralExpression.integrate(integrand, symbol)}{return Expression|nil}{integrand Expression, symbol SymbolExpression}

Recursive part of the indefinite integral operator; returns nil if the expression could not be integrated. 

\coderef{function IntegralExpression:isdefinite()}{return bool}

Returns \mintinline{lua}{true} of \texttt{IntegralExpression} is definite (i.e. if \texttt{.upper} and \texttt{.lower} are defined fields), otherwise returns \mintinline{lua}{false}. 


\end{document}