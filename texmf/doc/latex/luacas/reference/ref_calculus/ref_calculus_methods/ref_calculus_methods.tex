\documentclass{article}

\usepackage{luacas}
\usepackage{amsmath}
\usepackage{amssymb}

\usepackage[margin=1in]{geometry}
\usepackage[shortlabels]{enumitem}

\usepackage{pgfplots}
\pgfplotsset{compat=1.18}
\usetikzlibrary{positioning,calc}
\usepackage{forest}
\usepackage{minted}
\usemintedstyle{pastie}
\usepackage[hidelinks]{hyperref}
\usepackage{parskip}
\usepackage{multicol}
\usepackage[most]{tcolorbox}
    \tcbuselibrary{xparse,documentation}
\usepackage{microtype}
\usepackage{makeidx}
\usepackage{fontawesome5}

\usepackage[
backend=biber,
style=numeric,
]{biblatex}
\addbibresource{sources.bib}

\definecolor{rose}{RGB}{128,0,0}
\definecolor{roseyellow}{RGB}{222,205,99}
\definecolor{roseblue}{RGB}{167,188,214}
\definecolor{rosenavy}{RGB}{79,117,139}
\definecolor{roseorange}{RGB}{232,119,34}
\definecolor{rosegreen}{RGB}{61,68,30}
\definecolor{rosewhite}{RGB}{223,209,167}
\definecolor{rosebrown}{RGB}{108,87,27}
\definecolor{rosegray}{RGB}{84,88,90}

\definecolor{codegreen}{HTML}{49BE25}

\newtcolorbox{codebox}[1][sidebyside]{
    enhanced,skin=bicolor,
    #1,
    arc=1pt,
    colframe=brown,
    colback=brown!15,colbacklower=white,
    boxrule=1pt,
    notitle
}

\newtcolorbox{codehead}[1][]{
    enhanced,
    frame hidden,
    colback=rosegray!15,
    boxrule=0mm,
    leftrule=5mm,
    rightrule=5mm,
    boxsep=0mm,
    arc=0mm,
    outer arc=0mm,
    left=3mm,
    right=3mm,
    top=1mm,
    bottom=1mm,
    toptitle=1mm,
    bottomtitle=1mm,
    oversize,
    #1
}

\DeclareTotalTCBox{\lilcoderef}{O{} m m}{
    enhanced,
    frame hidden,
    colback=rosegray!15,
    enhanced,
    nobeforeafter,
    tcbox raise base,
    boxrule=0mm,
    leftrule=5mm,
    rightrule=5mm,
    boxsep=0mm,
    arc=0mm,
    outer arc=0mm,
    left=1mm,
    right=1mm,
    top=1mm,
    bottom=1mm,
    oversize,
    #1
}{\mintinline{lua}{#2} \mintinline{lua}{#3}}

\usepackage{varwidth}

\newtcolorbox{newcodehead}[2][]{
    enhanced,
    frame hidden,
    colback=rosegray!15,
    boxrule=0mm,
    leftrule=5mm,
    rightrule=5mm,
    boxsep=0mm,
    arc=0mm,
    outer arc=0mm,
    left=3mm,
    right=3mm,
    top=1mm,
    bottom=1mm,
    toptitle=1mm,
    bottomtitle=1mm,
    oversize,
    #1,
    fonttitle=\bfseries\ttfamily\footnotesize,
    coltitle=rosegray,
    attach boxed title to top text right,
    boxed title style={frame hidden,size=small,bottom=-1mm,
    interior style={fill=none,
    top color=white,
    bottom color=white}},
    title={#2}
}

\makeindex

\newcommand{\coderef}[2]{%
\begin{codehead}[sidebyside,segmentation hidden]%
    \mintinline{lua}{#1}%
    \tcblower%
    \begin{flushright}%
    \mintinline{lua}{#2}%
    \end{flushright}%
\end{codehead}%
}

\newcommand{\newcoderef}[3]{%
\begin{newcodehead}[sidebyside,segmentation hidden]{#3}%
    \mintinline{lua}{#1}%
    \tcblower%
    \begin{flushright}%
    \mintinline{lua}{#2}%
    \end{flushright}%
\end{newcodehead}%
}
\usepackage{marginnote}

\begin{document}
\setdescription{style=multiline,
        topsep=10pt,
        leftmargin=6.5cm,
        }

\subsection{Calculus Methods}

\newcoderef{function IntegralExpression.table(integral)}{return Expression|nil}{integral IntegralExpression}

Attempts to integrate \texttt{integral.expression} with respect to \texttt{integral.symbol} by checking a table of basic integrals; returns nil if the integrand isn't in the table. For example:

\begin{codebox}
    \begin{minted}[fontsize=\small]{latex}
\begin{CAS}
  vars('x')
  f = int(cos(x),x)
  f = f:table()
  g = int(x*cos(x),x)
  g = g:table()
\end{CAS}
\[ f = \print{f} \qquad g = \print{g} \] 
\end{minted}
\tcblower
\begin{CAS}
  vars('x')
  f = int(cos(x),x)
  f = f:table()
  g = int(x*cos(x),x)
  g = g:table()
  \end{CAS}
\[ f = \print{f} \qquad g = \print{g} \] 
\end{codebox}
The table of integrals consists of power functions, exponentials, logarithms, trigonometric, and inverse trigonometric functions. 

\newcoderef{function IntegralExpression.linearproperties(integral)}{return Expression|nil}{integral IntegralExpression}

Attempts to integrate \texttt{integral.expression} with respect to \texttt{integral.symbol} by using linearity properties (e.g. the integral of a sum/difference is the sum/difference of integrals); returns nil if any individual component cannot be integrated using \mintinline{lua}{IntegralExpression:integrate()}. For example:

\begin{codebox}
    \begin{minted}[fontsize=\small]{latex}
\begin{CAS}
  vars('x')
  f = int(sin(x) + e^x,x)
  g = f:table()
  f = f:linearproperties()
\end{CAS}
\[ f = \print*{f} \qquad g = \print*{g} \]
\end{minted}
\tcblower
\begin{CAS}
  f = int(sin(x) + e^x,x)
  g = f:table()
  f = f:linearproperties()
\end{CAS}
\[ f = \print*{f} \qquad g = \print*{g} \]
\end{codebox} 

\newcoderef{function IntegralExpression.substitutionmethod(integral)}{return Expression|nil}{integral IntegralExpression} 

Attempts to integrate \texttt{integral.expression} with respect to \texttt{integral.symbol} via $u$-substitution; returns nil if no suitable substitution is found to be successful. 

\begin{codebox}
    \begin{minted}[fontsize=\small]{latex}
\begin{CAS}
  vars('x')
  f = int(x*e^(x^2),x)
  g = int(x*e^x,x)
  f = f:substitutionmethod()
  g = g:substitutionmethod()
\end{CAS}
\[ f = \print*{f} \qquad g = \print*{g}.\] 
\end{minted}
\tcblower
\begin{CAS}
  vars('x')
  f = int(x*e^(x^2),x)
  g = int(x*e^x,x)
  f = f:substitutionmethod()
  g = g:substitutionmethod()
\end{CAS}
\[ f = \print*{f} \qquad g = \print*{g}.\] 
\end{codebox}

\newcoderef{function IntegralExpression.enhancedsubstitutionmethod(integral)}{return Expression|nil}{integral IntegralExpression} 

Attempts integrate \texttt{integral.expression} with respect to \texttt{integral.symbol} via $u$-substitutions. This method distinguishes itself from the \mintinline{lua}{.substitutionmethod} by attempted to solve $u= g(x)$ for the original variable and then substituting the result into the expression. This behavior is not included in \mintinline{lua}{.substitutionmethod} due to speed concerns. For example:

\begin{codebox}
    \begin{minted}[fontsize=\small]{latex}
\begin{CAS}
  vars('x')
  f = int(x^5*sqrt(x^3+1),x)
  g = f:substitutionmethod()
  h = f:enhancedsubstitutionmethod()
\end{CAS}
\[ g= \print*{g} \] 
\[ h= \print*{h} \] 
\end{minted}
\tcblower
\begin{CAS}
    vars('x')
    f = int(x^5*sqrt(x^3+1),x)
    g = f:substitutionmethod()
    h = f:enhancedsubstitutionmethod()
\end{CAS}
\[ g= \print*{g} \] 
\[ h= \print*{h} \] 
\end{codebox}


\newcoderef{function IntegralExpression.trialsubstitutions(Expression)}{return table<number, Expression}{}

Generates a list of possible $u$-substitutions to attempt in \texttt{substitutionmethod()} and \texttt{enhancedsubstitutionmethod()}. For example:

\begin{codebox}
    \begin{minted}[fontsize=\small]{latex}
\begin{CAS}
  vars('x')
  f = cos(x)/(1+sin(x))
  f = f:autosimplify()
  l = IntegralExpression.trialsubstitutions(f)
\end{CAS}
$\left\{ \lprint{l} \right\}$.
\end{minted}
\tcblower
\begin{CAS}
    vars('x')
    f = cos(x)/(1+sin(x))
    f = f:autosimplify()
    l = IntegralExpression.trialsubstitutions(f)
\end{CAS}
$ \left\{ \lprint{l} \right\}$.
\end{codebox}


\newcoderef{function IntegralExpression.rationalfunction(IntegralExpression)}{return Expression|nil}{}

Integrates \texttt{integrand} with respect to \texttt{symbol} via Lazard, Rioboo, Rothstein, and Trager's method in the case when \texttt{expression} is a rational function in the variable \texttt{symbol}. If \texttt{integrand} is not a rational function, then nil is returned. 

\begin{codebox}
    \begin{minted}[fontsize=\small]{latex}
\begin{CAS}
    vars('x')
    f = (x^2+2*x+2)/(x^2+3*x+2)
    f = f:autosimplify()
    g = int(f,x):rationalfunction()
\end{CAS}
\[ \int \print{f}\ dx = \print*{g} \] 
\end{minted}
\tcblower
\begin{CAS}
    vars('x')
    f = (x^2+2*x+2)/(x^2+3*x+2)
    f = f:autosimplify()
    g = int(f,x):rationalfunction()
\end{CAS}
\[ \int \print{f}\ dx = \print*{g} \] 
\end{codebox}

In some cases, the \mintinline{lua}{.rationalfunction} method returns non-standard results. For example:
\begin{codebox}
    \begin{minted}[fontsize=\small]{latex}
\begin{CAS}
  vars('x')
  num = x^2
  den = ((x+1)*(x^2+2*x+2)):expand()
  f = (num/den):autosimplify()
  f = int(f,x):rationalfunction()
\end{CAS}
\[ \print{simplify(f)} \] 
\end{minted}
\tcblower
\begin{CAS}
  vars('x')
  num = x^2
  den = ((x+1)*(x^2+2*x+2)):expand()
  f = (num/den):autosimplify()
  f = int(f,x):rationalfunction()
\end{CAS}
\[ \print{simplify(f)} \] 
\end{codebox}
On the other hand:
\begin{codebox}
    \begin{minted}[fontsize=\small]{latex}
\begin{CAS}
  pfrac = parfrac(num,den)
\end{CAS}
\[ \print*{int(pfrac,x)} \] 
\end{minted}
\tcblower
\begin{CAS}
    pfrac = parfrac(num,den)
  \end{CAS}
\[ \print*{int(pfrac,x)} \] 
\end{codebox}

\newcoderef{function IntegralExpression.partsmethod(IntegralExpression)}{return Expression|nil}{} 

Attempts to integrate \texttt{integral.expression} with respect to \texttt{integral.symbol} via \emph{integration by parts}; returns nil if no suitable application of IBP is found. For example:

\begin{codebox}
    \begin{minted}[fontsize=\small]{latex}
\begin{CAS}
    vars('x')
    a = int(x*e^x,x)
    b = a:partsmethod()
    c = int(e^(x^2),x)
    d = c:partsmethod()
\end{CAS}
\[ b=\print*{b} \] 
\[ d=\print*{d} \]
\end{minted}
\tcblower
\begin{CAS}
    vars('x')
    a = int(x*e^x,x)
    b = a:partsmethod()
    c = int(e^(x^2),x)
    d = c:partsmethod()
\end{CAS}
\[ b= \print*{b} \] 
\[ d= \print*{d} \] 
\end{codebox}


\newcoderef{function IntegralExpression.eulersformula(integral)}{return Expression|nil}{integral IntegralExpression}

Attempts to integrate \texttt{integral.expression} with respect to \texttt{integral.symbol} by using the Euler formulas: 
\[ \cos x = \frac{e^{ix} + e^{-ix}}{2} \qquad \sin x = \frac{e^{ix} - e^{-ix}}{2i}.\]  
Per usual, this method returns nil if such a method is unsuccessful (or if the integrand is unchanged after applying the above substitutions). This can often be used as an alternative for integration by parts. For example:

\begin{codebox}
    \begin{minted}[fontsize=\small]{latex}
\begin{CAS}
  vars('x')
  a = int(e^x*sin(x),x)
  b = int(x^2,x)
  c = a:eulersformula()
  d = b:eulersformula()
\end{CAS}
\[ c= \print*{c} \] 
\[ d= \print*{d} \]
\end{minted}
\tcblower
\begin{CAS}
    vars('x')
    a = int(e^x*sin(x),x)
    b = int(x^2,x)
    c = a:eulersformula()
    d = b:eulersformula()
\end{CAS}
\[ c= \print*{c} \] 
\[ d= \print*{d} \]
\end{codebox}

\newcoderef{function IntegralExpression.integrate(integral)}{return Expression|nil}{integral IntegralExpression}

Recursive part of the indefinite integral operator; returns nil if the expression could not be integrated. The methods above get called (roughly) in the following order:
\begin{enumerate}[(i)]
    \item \mintinline{lua}{.table}
    \item \mintinline{lua}{.linearproperties}
    \item \mintinline{lua}{.substitutionmethod}
    \item \mintinline{lua}{.rationalfunction}
    \item \mintinline{lua}{.partsmethod} 
    \item \mintinline{lua}{.eulersformula} 
    \item \mintinline{lua}{.enhancedsubstitutionmethod}
\end{enumerate}
Between (vi) and (vii), the \mintinline{lua}{.integrate} method will attempt to expand the integrand and retry. The method is recursive in the sense that (most) of the methods listed above will call \mintinline{lua}{.integrate} at some point. For example, after a list of trial substitutions is created, the method \mintinline{lua}{.substitutionmethod} will call \mintinline{lua}{.integrate} to determine whether the new integrand can be integrated via the methods in the above list. 

\subsubsection*{Parsing}

Recall the function \mintinline{lua}{int()} which acts as a shortcut for \mintinline{lua}{IntegralExpression:new()}. When \mintinline{lua}{:autosimplify()} is called upon an \texttt{IntegralExpression}, then \mintinline{lua}{IntegralExpression.integrate} is applied. If \mintinline{lua}{nil} is returned, then \mintinline{lua}{:autosimplify()} returns \mintinline{lua}{self}; otherwise the result of \mintinline{lua}{.integrate} is returned and evaluated over the bounds, if any are given. For example:

\begin{codebox}
    \begin{minted}[fontsize=\small]{latex}
\begin{CAS}
    vars('x')
    f = cos(x)*e^(sin(x))
    f = int(f,x,0,pi/2)
\end{CAS}
\[ \print{f} = \print*{f}\] 
\end{minted}
\tcblower
\begin{CAS}
    vars('x')
    f = cos(x)*e^(sin(x))
    f = int(f,x,0,pi/2)
\end{CAS}
\[ \print{f} = \print*{f}\] 
\end{codebox}

On the other hand:
\begin{codebox}
    \begin{minted}[fontsize=\small]{latex}
\begin{CAS}
    vars('x')
    f = e^(e^x)
    f = int(f,x,0,1)
\end{CAS}
\[ \print{f} = \print*{f} \] 
\end{minted}
\tcblower
\begin{CAS}
    vars('x')
    f = e^(e^x)
    f = int(f,x,0,1)
\end{CAS}
\[ \print{f} = \print*{f} \] 
\end{codebox}

\coderef{function IntegralExpression:isdefinite()}{return bool}

Returns \mintinline{lua}{true} of \texttt{IntegralExpression} is definite (i.e. if \texttt{.upper} and \texttt{.lower} are defined fields), otherwise returns \mintinline{lua}{false}. 


\end{document}