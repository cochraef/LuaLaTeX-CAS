\documentclass{article}

\usepackage{luacas}
\usepackage{amsmath}
\usepackage{amssymb}

\usepackage[margin=1in]{geometry}
\usepackage[shortlabels]{enumitem}

\usepackage{pgfplots}
\pgfplotsset{compat=1.18}
\usetikzlibrary{positioning,calc}
\usepackage{forest}
\usepackage{minted}
\usemintedstyle{pastie}
\usepackage[hidelinks]{hyperref}
\usepackage{parskip}
\usepackage{multicol}
\usepackage[most]{tcolorbox}
    \tcbuselibrary{xparse,documentation}
\usepackage{microtype}
\usepackage{makeidx}
\usepackage{fontawesome}

\usepackage[
backend=biber,
style=numeric,
]{biblatex}
\addbibresource{sources.bib}

\definecolor{rose}{RGB}{128,0,0}
\definecolor{roseyellow}{RGB}{222,205,99}
\definecolor{roseblue}{RGB}{167,188,214}
\definecolor{rosenavy}{RGB}{79,117,139}
\definecolor{roseorange}{RGB}{232,119,34}
\definecolor{rosegreen}{RGB}{61,68,30}
\definecolor{rosewhite}{RGB}{223,209,167}
\definecolor{rosebrown}{RGB}{108,87,27}
\definecolor{rosegray}{RGB}{84,88,90}

\definecolor{codegreen}{HTML}{49BE25}

\newtcolorbox{codebox}[1][sidebyside]{
    enhanced,skin=bicolor,
    #1,
    arc=1pt,
    colframe=brown,
    colback=brown!15,colbacklower=white,
    boxrule=1pt,
    notitle
}

\newtcolorbox{codehead}[1][]{
    enhanced,
    frame hidden,
    colback=rosegray!15,
    boxrule=0mm,
    leftrule=5mm,
    rightrule=5mm,
    boxsep=0mm,
    arc=0mm,
    outer arc=0mm,
    left=3mm,
    right=3mm,
    top=1mm,
    bottom=1mm,
    toptitle=1mm,
    bottomtitle=1mm,
    oversize,
    #1
}

\usepackage{varwidth}

\newtcolorbox{newcodehead}[2][]{
    enhanced,
    frame hidden,
    colback=rosegray!15,
    boxrule=0mm,
    leftrule=5mm,
    rightrule=5mm,
    boxsep=0mm,
    arc=0mm,
    outer arc=0mm,
    left=3mm,
    right=3mm,
    top=1mm,
    bottom=1mm,
    toptitle=1mm,
    bottomtitle=1mm,
    oversize,
    #1,
    fonttitle=\bfseries\ttfamily\footnotesize,
    coltitle=rosegray,
    attach boxed title to top text right,
    boxed title style={frame hidden,size=small,bottom=-1mm,
    interior style={fill=none,
    top color=white,
    bottom color=white}},
    title={#2}
}

\makeindex

\newcommand{\coderef}[2]{%
\begin{codehead}[sidebyside,segmentation hidden]%
    \mintinline{lua}{#1}%
    \tcblower%
    \begin{flushright}%
    \mintinline{lua}{#2}%
    \end{flushright}%
\end{codehead}%
}

\newcommand{\newcoderef}[3]{%
\begin{newcodehead}[sidebyside,segmentation hidden]{#3}%
    \mintinline{lua}{#1}%
    \tcblower%
    \begin{flushright}%
    \mintinline{lua}{#2}%
    \end{flushright}%
\end{newcodehead}%
}

\begin{document}
\setdescription{style=multiline,
        topsep=10pt,
        leftmargin=5cm,
        }

\subsection{Algebra Classes}

The algebra package contains functionality for arbitrary-precision arithmetic, polynomial arithmetic and factoring, symbolic root finding, and logarithm and trigonometric expression classes. It requires the core package to be loaded.

The abstract classes in the algebra module all inherit from the \texttt{ConstantExpression} branch in the inheritance tree:

\begin{itemize}
    \item \texttt{Ring}
    \item \texttt{EuclideanDomain}
    \item \texttt{Field}
\end{itemize}    

The {\ttfamily EuclideanDomain} class is a sub-class to the {\ttfamily Ring} class, and the {\ttfamily Field} class is a sub-class to the {\ttfamily EuclideanDomain} class.

The following concrete classes inherit from the {\ttfamily Ring} class (or one of the sub-classes mentioned above). However, not all of them are proper {\ttfamily ConstantExpression}s, so some of them override the {\ttfamily isconstant()} method.

\begin{itemize}
    \item {\ttfamily Integer} 
    \item {\ttfamily IntegerModN}
    \item {\ttfamily Rational}
    \item {\ttfamily PolynomialRing}
\end{itemize}

The other concrete classes in the Algebra package do not inherit from the {\ttfamily Ring} interface, instead they inherit from the {\ttfamily CompoundExpression} interface:

\begin{itemize}
    \item {\ttfamily AbsExpression}
    \item {\ttfamily Logarithm}
    \item {\ttfamily FactorialExpression}
    \item {\ttfamily SqrtExpression}
    \item {\ttfamily TrigExpression}
    \item {\ttfamily RootExpression}
\end{itemize}

\newcoderef{function Integer:new(n)}{return Integer}{n number|string|Integer}
\index{Algebra!Classes!\texttt{SymbolExpression}}
\addcontentsline{toc}{subsubsection}{\ttfamily Integer}

Takes a string, number, or {\ttfamily Integer} input and constructs an \texttt{Integer}. Since Lua can only store integers exactly up to a certain point, it is recommended to use strings to build large integers.

\begin{codebox}
    \begin{minted}[fontsize=\small]{lua}
a = Integer(-12435)
b = Integer('-12435')
tex.print('\\[',a:tolatex(),
    '=',
    b:tolatex(),
    '\\]')
\end{minted}
\tcblower
\directlua{
    a = Integer(-12435)
    b = Integer('-12435')
    tex.print('\\[',a:tolatex(),
        '=',
        b:tolatex(),
        '\\]')
}
\end{codebox}
An {\ttfamily Integer} is a table 1-indexed by Lua numbers consisting of Lua numbers. For example:
\begin{codebox}
    \begin{minted}[fontsize=\small]{lua}
tex.print(tostring(b[1]))
\end{minted}
\tcblower
\directlua{
    tex.print(tostring(b[1]))
}
\end{codebox}
Whereas:
\begin{codebox}[]
    \begin{minted}[fontsize=\small]{lua}
c = Integer('7240531360949381947528131508')
tex.print('The first 14 digits of c:', tostring(c[1]),'. ')
tex.print('The last 14 digits of c:', tostring([2]),'.')
\end{minted}
\tcblower
\directlua{
    c = Integer('7240531360949381947528131508')
    tex.print('The first 14 digits of c:', tostring(c[1]),'. ')
    tex.print('The last 14 digits of c:', tostring(c[2]),'.')
}
\end{codebox}

The global field {\ttfamily DIGITSIZE} is set to \texttt{14} so that exact arithmetic on {\ttfamily Integer}s can be done as efficiently as possible while respecting Lua's limitations.

\subsubsection*{Fields}
{\ttfamily Integer}s have a {\ttfamily .sign} field which contains the Lua number {\ttfamily 1} or {\ttfamily -1} depending on whether \texttt{Integer} is positive or negative. 
\begin{codebox}[]
    \begin{minted}[fontsize=\small]{lua}
tex.print('The sign of',tostring(b),'is:',tostring(b.sign))
\end{minted}
\tcblower
\directlua{
    tex.print('The sign of',
        tostring(b),
        'is:',
        tostring(b.sign))
}
\end{codebox}

\subsubsection*{Parsing}

The contents of the environment \mintinline{latex}{\begin{CAS}..\end{CAS}} are wrapped in the argument of a function \mintinline{lua}{CASparse()} which, among other things, seeks out digit strings intended to represent integers, and wraps those in \texttt{Integer('...')}. 

\begin{codebox}
    \begin{minted}[fontsize=\small]{latex}
\begin{CAS}
    c = 7240531360949381947528131508
\end{CAS}
\directlua{
    tex.print(tostring(c[1]))
} 
\end{minted}
\tcblower
\begin{CAS}
    c = 7240531360949381947528131508
\end{CAS}
\directlua{
    tex.print(tostring(c[1]))
} 
\end{codebox}

\newcoderef{function IntegerModN:new(i,n)}{return IntegerModN}{i Integer, n Integer}
\index{Algebra!Classes!\texttt{IntegerModN}}
\addcontentsline{toc}{subsubsection}{\ttfamily IntegerModN}

Takes an {\ttfamily Integer i} and {\ttfamily Integer n} and constructs an integer in the ring $\mathbb{Z}_n$, the integers modulo $n$. 

\begin{codebox}[]
    \begin{minted}[fontsize=\small]{lua}
i = Integer(143)
n = Integer(57)
a = IntegerModN(i,n)
tex.print('\\[',i:tolatex(),'\\equiv',a:tolatex(),'\\]')
\end{minted}
\tcblower
\luaexec{
    i = Integer(143)
    n = Integer(57)
    a = IntegerModN(i,n)
    tex.print('\\[',i:tolatex(),'\\equiv',a:tolatex(),'\\]')
}
\end{codebox}

\subsubsection*{Fields}

{\ttfamily IntegerModN}s have two fields: {\ttfamily .element} and {\ttfamily .modulus}. The reduced input \texttt{i} is stored in {\ttfamily .element} while the input \texttt{n} is stored in {\ttfamily .modulus}:

\begin{codebox}
    \begin{minted}[fontsize=\small]{lua}
tex.print(a.element:tolatex(),'\\newline')
tex.print(a.modulus:tolatex())
\end{minted}
\tcblower
\luaexec{
    tex.print(a.element:tolatex(),'\\newline')
    tex.print(a.modulus:tolatex())
}
\end{codebox}

\subsubsection*{Parsing}

The function \texttt{Mod(,)} is a shortcut for \texttt{IntegerModN(,)}:
\begin{codebox}
    \begin{minted}[fontsize=\small]{latex}
\begin{CAS}
    i = 143
    n = 57
    a = Mod(i,n)
\end{CAS}
\[ \print{i} \equiv \print{a} \] 
\end{minted}
\tcblower
\begin{CAS}
    i = 143
    n = 57
    a = Mod(i,n)
\end{CAS}
\[ \print{i} \equiv \print{a} \] 
\end{codebox}

\newcoderef{function PolynomialRing:new(coefficients, symbol, degree)}{return PolynomialRing}{coefficients table<number,Ring>, symbol string|SymbolExpression, degree Integer}
\index{Algebra!Classes!\texttt{PolynomialRing}}
\addcontentsline{toc}{subsubsection}{\ttfamily PolynomialRing}

Takes a table of {\ttfamily coefficients} not all necessarily in the same ring and a {\ttfamily symbol} to create a polynomial with the given {\ttfamily symbol} and coefficients in the smallest possible ring that contains all of the coefficients. If {\ttfamily degree} is omitted, it will calculate the degree of the polynomial automatically. The list can either be one-indexed or zero-indexed, but if it is one-indexed, the internal list of coefficients will still be zero-indexed.

\begin{codebox}
    \begin{minted}[fontsize=\small]{latex}
\begin{CAS}
  f = PolynomialRing({0,1/3,-1/2,1/6},'t')
\end{CAS}
\[ \print{f} \] 
\end{minted}
\tcblower
\begin{CAS}
    f = PolynomialRing({0,1/3,-1/2,1/6},'t')
\end{CAS}
\[ \print{f} \] 
\end{codebox}
The \texttt{PolynomialRing} class overwrites the \mintinline{lua}{isatomic()} and \mintinline{lua}{isconstant()} inheritances from the abstract class \texttt{ConstantExpression}. 
\subsubsection*{Fields}

\begin{multicols}{2}
{\ttfamily PolynomialRing}s have several fields:
\begin{itemize}
    \item {\ttfamily f.coefficients} stores the 0-indexed table of coefficients of {\ttfamily f};
    \item {\ttfamily f.degree} stores the {\ttfamily Integer} that represents the degree of {\ttfamily f};
    \item {\ttfamily f.symbol} stores the {\ttfamily string} representing the variable or {\ttfamily symbol} of {\ttfamily f}.
    \item {\ttfamily f.ring} stores the \texttt{RingIdentifier} for the ring of coefficients.
\end{itemize}

\columnbreak

\parseshrub{f}
\bracketset{action character = @}
\begin{center}
\begin{forest}
    for tree = {font = \ttfamily,
        draw,
        rounded corners = 1pt,
        fill=gray!20,
        s sep = 1.5cm,
        l sep = 2cm}
    @\shrubresult
\end{forest}
\end{center}
\end{multicols}
For example:
\begin{codebox}
    \begin{minted}[fontsize=\small]{lua}
for i=0,f.degree:asnumber() do
  tex.print('\\[',
    f.coefficients[i]:tolatex(),
    f.symbol,
    '^{',
    tostring(i),
    '}\\]')
end
if f.ring == Rational.getring() then 
  tex.print('Rational coefficients')
end
\end{minted}
\tcblower
\luaexec{
for i=0,f.degree:asnumber() do
  tex.print(
    '\\[',
    f.coefficients[i]:tolatex(),
    f.symbol,
    '^{',
    tostring(i),
    '}\\]'
  )
  end
  if f.ring == Rational.getring() then 
    tex.print('Rational coefficients')
  end
}
\end{codebox}

\subsubsection*{Parsing}

The function \mintinline{lua}{Poly()} is a shortcut for \mintinline{lua}{PolynomialRing:new()}. If the second argument \texttt{symbol} is omitted, then the default is \texttt{'x'}:

\begin{codebox}
    \begin{minted}[fontsize=\small]{latex}
\begin{CAS}
    f = Poly({0,1/3,-1/2,1/6})
\end{CAS}
\[ \print{f} \] 
\end{minted}
\tcblower
\begin{CAS}
    f = Poly({0,1/3,-1/2,1/6})
\end{CAS}
\[ \print{f} \] 
\end{codebox}

Alternatively, one could typeset the polynomial naturally and use the \texttt{topoly()} function. This is the same as the \texttt{topolynomial()} method except that the \texttt{autosimplify()} method is automatically called first:

\begin{codebox}
    \begin{minted}[fontsize=\small]{latex}
\begin{CAS}
    vars('x')
    f = 1/3*x - 1/2*x^2 + 1/6*x^3
    f = topoly(f)
\end{CAS}
\[ \print{f} \] 
\end{minted}
\tcblower
\begin{CAS}
    vars('x')
    f = 1/3*x - 1/2*x^2 + 1/6*x^3
    f = topoly(f)
\end{CAS}
\[ \print{f} \] 
\end{codebox}

\newcoderef{function Rational:new(n,d,keep)}{return Rational}{n Ring, d Ring, keep bool}
\index{Algebra!Classes!\texttt{Rational}}
\addcontentsline{toc}{subsubsection}{\ttfamily Rational}

Takes a numerator {\ttfamily n} and denominator {\ttfamily d} in the same {\ttfamily Ring} and constructs a rational expression in the field of fractions over that ring. For the integers, this is the ring of rational numbers. If the {\ttfamily keep} flag is omitted, the constructed object will be simplified to have smallest possible denominator, possibly returning an object in the original {\ttfamily Ring}. Typically, the {\ttfamily Ring} will be either {\ttfamily Integer} or {\ttfamily PolynomialRing}, so {\ttfamily Rational} can be viewed as a constructor for either a rational number or a rational function. 

For example:
\begin{codebox}
    \begin{minted}[fontsize=\small]{lua}
a = Integer(6)
b = Integer(10)
c = Rational(a,b)
tex.print('\\[',c:tolatex(),'\\]')
\end{minted}
\tcblower
\luaexec{
    a = Integer(6)
    b = Integer(10)
    c = Rational(a,b)
    tex.print('\\[',c:tolatex(),'\\]')
}
\end{codebox}
But also:
\begin{codebox}
    \begin{minted}{lua}
a = Poly({Integer(2),Integer(3)})
b = Poly({Integer(4),Integer(1)})
c = Rational(a,b)
tex.print('\\[',c:tolatex(),'\\]')
\end{minted}
\tcblower
\luaexec{
a = Poly({Integer(2),Integer(3)})
b = Poly({Integer(4),Integer(1)})
c = Rational(a,b)
tex.print('\\[',c:tolatex(),'\\]')
}
\end{codebox}

\subsubsection*{Fields}

\texttt{Rational}s naturally have the two fields: \texttt{numerator}, \texttt{denominator}. These fields store precisely what you think.

If \texttt{numerator} or \texttt{denominator} are \texttt{PolynomialRing}s, then the constructed \texttt{Rational} will have two additional fields: \texttt{symbol} and \texttt{ring}. The field \texttt{ring} stores the \texttt{RingIdentifier} for the \texttt{PolynomialRing} to which either the numerator or denominator belong.

\begin{codebox}[]
    \begin{minted}{lua}
if c.ring == PolynomialRing.getring() then 
  tex.print('$',c:tolatex(),'$ is a Rational Function in the variable',c.symbol)
end
\end{minted}
\tcblower
\luaexec{
if c.ring == PolynomialRing.getring() then 
  tex.print('$',c:tolatex(),'$ is a Rational Function in the variable',c.symbol)
end
}
\end{codebox}
If the \texttt{.ring} field is nil, then the \texttt{Rational} represents a rational number (i.e., a ratio of \texttt{Integer}s). 

\subsubsection*{Parsing}

\texttt{Raional}s are constructed naturally using the \texttt{/} operator:

\begin{codebox}
    \begin{minted}[fontsize=\small]{latex}
\begin{CAS}
    a = Poly({2,3})
    b = Poly({4,1})
    c = a/b
\end{CAS}
\[ \print{c} \] 
\end{minted}
\tcblower
\begin{CAS}
    a = Poly({2,3})
    b = Poly({4,1})
    c = a/b
\end{CAS}
\[ \print{c} \] 
\end{codebox}

\coderef{function AbsExpression:new(expression)}{return AbsExpression}

Creates a new absolute value expression with the given expression.

\begin{codebox}
    \begin{minted}[fontsize=\small]{latex}
\begin{CAS}
    f = Poly({1,1})
    g = Poly({-1,1})
    h = AbsExpression(f/g)
\end{CAS}
\[ h = \print{h} \] 
\end{minted}
\tcblower
\begin{CAS}
    f = Poly({1,1})
    g = Poly({-1,1})
    h = AbsExpression(f/g)
\end{CAS}
\[ h = \print{h} \] 
\end{codebox}

\subsubsection*{Fields}

\texttt{AbsExpression}s have only one field: \texttt{.expression}. This field simply holds the \texttt{Expression} inside the absolute value:
\begin{multicols}{2}
\begin{codebox}[]
\begin{minted}[fontsize=\small]{lua}
tex.print('\\[',
    h.expression:tolatex(),
    '\\]')
\end{minted}
\tcblower
\directlua{
    tex.print('\\[',h.expression:tolatex(),'\\]')
}
\end{codebox}
\parseshrub{h}
\bracketset{action character = @}
\begin{center}
\begin{forest}
    for tree = {font=\ttfamily,
        draw,
        rounded corners=1pt,
        fill=gray!20,
        l sep =1.5cm}
    @\shrubresult
\end{forest}
\end{center}
\end{multicols}

\subsubsection*{Parsing}

The function \mintinline{lua}{abs()} is a shortcut to \mintinline{lua}{AbsExpression:new()}. For example:

\begin{codebox}
    \begin{minted}[fontsize=\small]{latex}
\begin{CAS}
    f = Poly({1,1})
    g = Poly({-1,1})
    h = abs(f/g)
\end{CAS}
\[ h = \print{h} \] 
\end{minted}
\tcblower
\begin{CAS}
    f = Poly({1,1})
    g = Poly({-1,1})
    h = abs(f/g)
\end{CAS}
\[ h = \print{h} \] 
\end{codebox}

\newcoderef{function Logarithm:new(base,arg)}{return Logarithm}{base Expression, arg Expression}

Creates a new \texttt{Logarithm} expression with the given \texttt{base} and \texttt{arg}ument. Some basic simplification rules are known to \texttt{autosimplify()}:

\begin{codebox}
    \begin{minted}[fontsize=\small]{latex}
\begin{CAS}
    vars('b','x','y')
    f = Logarithm(b,x^y)
\end{CAS}
\[ \print{f} = \print*{f} \] 
\end{minted}
\tcblower
\begin{CAS}
    vars('b','x','y')
    f = Logarithm(b,x^y)
\end{CAS}
\[ \print{f} = \print*{f} \] 
\end{codebox}

\subsubsection*{Fields}

\begin{multicols}{2}
\texttt{Logarithm}s have two fields: \texttt{base} and \texttt{expression}; \texttt{base} naturally stores the base of the logarithm (i.e., the first argument of \texttt{Logarithm}) while \texttt{expression} stores the argument of the logarithm (i.e., the second argument of \texttt{Logarithm}). 

\begin{center}
    \parseshrub{f}
    \bracketset{action character = @}
    \begin{forest}
        for tree = {font = \ttfamily,
            draw,
            rounded corners=1pt,
            fill = gray!20,
            s sep = 1.5cm}
        @\shrubresult 
    \end{forest}
\end{center}
\end{multicols}

\subsubsection*{Parsing}

The function \mintinline{lua}{log()} is a shortcut to \texttt{Logarithm}:

\begin{codebox}
    \begin{minted}[fontsize=\small]{latex}
\begin{CAS}
    vars('b')
    f = log(b,b)
\end{CAS}
\[ \print{f} = \print*{f} \] 
\end{minted}
\tcblower
\begin{CAS}
    vars('b')
    f = log(b,b)
\end{CAS}
\[ \print{f} = \print*{f} \] 
\end{codebox}

There is also a \mintinline{lua}{ln()} function to shortcut \texttt{Logarithm} where the base is \texttt{e}, the natural exponent.

\begin{codebox}
    \begin{minted}[fontsize=\small]{latex}
\begin{CAS}
    f = ln(e)
\end{CAS}
\[ \print{f} = \print*{f} \] 
\end{minted}
\tcblower
\begin{CAS}
    f = ln(e)
\end{CAS}
\[ \print{f} = \print*{f} \] 
\end{codebox}

\newcoderef{function FactorialExpression:new(expression)}{return FactorialExpression}{expression Expression}

Creates a new \texttt{FactorialExpression} with the given \texttt{expression}. For example:

\begin{codebox}
    \begin{minted}[fontsize=\small]{latex}
\begin{CAS}
    a = FactorialExpression(5)
\end{CAS}
\[ \print{a} \] 
\end{minted}
\tcblower
\begin{CAS}
    a = FactorialExpression(5)
\end{CAS}
\[ \print{a} \] 
\end{codebox}
The \texttt{evaluate()} method will compute factorials of positive \texttt{Integer}s:

\begin{codebox}
    \begin{minted}[fontsize=\small]{latex}
\begin{CAS}
    a = FactorialExpression(5)
\end{CAS}
\[ \print{a} = \print{a:evaluate()} \] 
\end{minted}
\tcblower
\begin{CAS}
    a = FactorialExpression(5)
\end{CAS}
\[ \print{a} = \print{a:evaluate()} \] 
\end{codebox}

\subsubsection*{Fields}

\texttt{FactorialExpression}s have only one field: \texttt{expression}. This field stores the argument of \texttt{FactorialExpression()}. 

\subsubsection*{Parsing}

The function \mintinline{lua}{factorial()} is a shortcut to \texttt{FactorialExpression()}:

\begin{codebox}
    \begin{minted}[fontsize=\small]{latex}
\begin{CAS}
    a = factorial(5)
\end{CAS}
\[ \print{a} = \print{a:evaluate()} \] 
\end{minted}
\tcblower
\begin{CAS}
    a = factorial(5)
\end{CAS}
\[ \print{a} = \print{a:evaluate()} \] 
\end{codebox}

\newcoderef{function SqrtExpression:new(expression, root)}{return SqrtExpression}{expression Expression, root Integer}

Creates a new \texttt{SqrtExpression} with the given \texttt{expression} and \texttt{root}. If \texttt{root} is omitted, then \texttt{root} defaults to \mintinline{lua}{Integer(2)}. For example:

\begin{codebox}
    \begin{minted}[fontsize=\small]{lua}
a = SqrtExpression(Integer(8))
b = SqrtExpression(Integer(8),Integer(3))
c = a+b 
tex.print('\\[',c:tolatex(),'\\]')
\end{minted}
\tcblower
\directlua{
    a = SqrtExpression(Integer(8))
b = SqrtExpression(Integer(8),Integer(3))
c = a+b 
tex.print('\\[',c:tolatex(),'\\]')
}
\end{codebox}
The \texttt{autosimplify()} method does a few things here. For the example above:
\begin{codebox}
    \begin{minted}[fontsize=\small]{lua}
a = SqrtExpression(Integer(8))
b = SqrtExpression(Integer(8),Integer(3))
c = (a+b):autosimplify() 
tex.print('\\[',c:tolatex(),'\\]')
\end{minted}
\tcblower
\directlua{
    a = SqrtExpression(Integer(8))
b = SqrtExpression(Integer(8),Integer(3))
c = (a+b):autosimplify() 
tex.print('\\[',c:tolatex(),'\\]')
}
\end{codebox}
On the other hand, if \texttt{root} or \texttt{expression} are not constants, then typically \mintinline{lua}{autosimplify()} will convert \texttt{SqrtExpression} to the appropriate \texttt{BinaryOperation}. For example:

\directlua{
    vars('x')
    a = SqrtExpression(x,Integer(3))
    b = a:autosimplify()
}

\begin{multicols}{2}
    \begin{center}
        \underline{Tree for \texttt{a}}

\parseshrub{a}
\bracketset{action character = @}
\begin{forest}
    for tree = {s sep=2cm,
        font=\ttfamily,
        draw,
        rounded corners = 1pt,
        fill=gray!20}
    @\shrubresult
\end{forest}

        \underline{Tree for \texttt{a:autosimplify()}}

\parseshrub{a:autosimplify()}
\bracketset{action character = @}
\begin{forest}
    for tree = {s sep=2cm,
        font=\ttfamily}
    @\shrubresult
\end{forest}
\end{center}
\end{multicols}

\subsubsection*{Parsing}

The function \mintinline{lua}{sqrt()} shortcuts \texttt{SqrtExpression()}:

\begin{codebox}
    \begin{minted}[fontsize=\small]{latex}
\begin{CAS}
    a = sqrt(1/9)
    b = sqrt(27/16,3)
    c = a+b
\end{CAS}
\[ \print{c} = \print*{c} \] 
\end{minted}
\tcblower
\begin{CAS}
    a = sqrt(1/9)
    b = sqrt(27/16,3)
    c = a+b
\end{CAS}
\[ \print{c} = \print*{c} \] 
\end{codebox}

\newcoderef{function TrigExpression:new(name,expression)}{return TrigExpression}{name string|SymbolExpression, expression Expression}

Creates a new trig expression with the given \texttt{name} and \texttt{expression}. For example:

\begin{codebox}
    \begin{minted}[fontsize=\small]{lua}
vars('x')
f = TrigExpression('sin',x)
tex.print('\\[',f:tolatex(),'\\]')
\end{minted}
\tcblower
\directlua{
    vars('x')
    f = TrigExpression('sin',x)
    tex.print('\\[',f:tolatex(),'\\]')
}
\end{codebox}

\subsubsection*{Fields}

\begin{multicols}{2}

\texttt{TrigExpression}s have many fields:
\begin{itemize}
    \item \mintinline{lua}{TrigExpression.name} stores the string \texttt{name}, i.e. the first argument of \mintinline{lua}{TrigExpression()};
    \item \mintinline{lua}{TrigExpression.expression} stores the \texttt{Expression} \texttt{expression}, i.e. the second argument of \mintinline{lua}{TrigExpression()};
    \item and all fields inherited from \texttt{FunctionExpression} (e.g. \mintinline{lua}{TrigExpression.derivatives} which defaults to \mintinline{lua}{Integer.zero()}). 
\end{itemize}

\columnbreak

\begin{center}
\parseshrub{f}
\bracketset{action character = @}
\begin{forest}
    for tree = {font = \ttfamily,
        draw,
        rounded corners = 1pt,
        fill = gray!20,
        l sep = 2cm}
    @\shrubresult
\end{forest}
\end{center}
\end{multicols}

\subsubsection*{Parsing}

The usual trigonometric functions have the anticipated shortcut names. For example:

\begin{codebox}
    \begin{minted}[fontsize=\small]{latex}
\begin{CAS}
    f = arctan(x^2)
\end{CAS}
\[ \print{f} \] 
\end{minted}
\tcblower
\begin{CAS}
    f = arctan(x^2)
\end{CAS}
\[ \print{f} \] 
\end{codebox}

\end{document}