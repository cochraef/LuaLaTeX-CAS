\documentclass{article}
\usepackage{fullpage}
\usepackage{luacas}
\usepackage{forest}

\begin{document}
Hi 
\begin{CAS}
    vars('x','y')
    f = x^3
\end{CAS}
\[\print{f} \]

\luaexec{
    results = {}
    for index, expression in ipairs(f:subexpressions()) do
        results[index] = expression:expand()
    end
    expanded = BinaryOperation(f.operation, results)
}
\[ \print{expanded} \] 
\parseforest{expanded}
\bracketset{action character = @}
\begin{forest}
    @\forestresult
\end{forest}

\luaexec{
    if expanded.operation == BinaryOperation.POW and expanded.expressions[2]:type() == Integer then
        exp = BinaryOperation.MULEXP({Integer.one()});
        pow = expanded.expressions[2]:asnumber()
        for _ = 1, math.abs(pow) do
            exp = exp:expand2(expanded.expressions[1])
            if _ > 1 then 
                exp = exp:autosimplify()
            end
            tex.print('$',exp:tolatex(),'$')
        end
    end
}

\luaexec{
    exp = x^Integer(2)
    exp = exp:expand2(x)
    tex.print('$',exp:tolatex(),'$')
}

\begin{CAS}
    g = x^2*y^3
    g = g^2
    g = g:expand()
    f = x^3
    f = f:expand()
\end{CAS}
\[ \print*{g} \qquad \print{f}\] 

\begin{CAS}
    g = (x^2+1)*(x^2+2*x+2)
    den = g:expand():topolynomial()
    num = Poly({1})
    pfrac = PolynomialRing.partialfractions(num,den)   
\end{CAS}
\[ \print{pfrac}\] 
\[ \print*{int(pfrac,x)} \] 

\end{document}

\begin{CAS}
    vars('x')
    f = 1/(x^4+2*x^2+1)
    f = f:autosimplify()
\end{CAS}
\[ \print{f} \] 
\parseforest{f}
\bracketset{action character = @}
\begin{forest}
    @\forestresult
\end{forest}

\luaexec{
    den,dencheck = f.expressions[1]:topolynomial()
    if dencheck then 
        tex.print('pass')
    end
    num = PolynomialRing({Integer.one()}, den.symbol)
}
\[ \print{num/den} \] 
\luaexec{
    gcd = PolynomialRing.gcd(num,den)
    if gcd ~= Integer.one() then 
        f,g = f // gcd, g // gcd 
    end 
    if gcd == Integer.one() then 
        tex.print('pass')
    end
}
\luaexec{
    q,r = num:divremainder(den)
    U = IntegralExpression.integrate(q,x)
    sfden = den:squarefreefactorization()
    pfrac = PolynomialRing.partialfractions(r,den,sfden)
}
\[ \text{pfrac} = \print{pfrac} \] 
\luaexec{
    V = Integer.zero()
    for _, term in ipairs(pfrac.expressions) do
        local i = \#term.expressions
        if i > 1 then
            for j = 1, i-1 do
                local n = term.expressions[j].expressions[1]
                local d = term.expressions[j].expressions[2].expressions[1]
                local p = term.expressions[j].expressions[2].expressions[2]
                local _, s, t = PolynomialRing.extendedgcd(d, d:derivative())
                s = s * n
                t = t * n
                V = V - t / ((p-Integer.one()) * BinaryOperation.POWEXP({d, p-Integer.one()}))
                term.expressions[j+1].expressions[1] = term.expressions[j+1].expressions[1] + s + t:derivative() / (p-Integer.one())
            end
        end
    end
}
\[ \text{V} = \print{V} \] 
\luaexec{
    local W = Integer.zero()
    for _, term in ipairs(pfrac.expressions) do
        a = term.expressions[\#term.expressions].expressions[1]
        tex.print('\\[', a:tolatex(),'\\]')
        b = term.expressions[1].expressions[2].expressions[1]
        tex.print('\\[', b:tolatex(),'\\]')
        y = a - b:derivative() * PolynomialRing({Integer.zero(), Integer.one()}, "z")
        tex.print('\\[',y:tolatex(),'\\]')
        if b.ring == PolynomialRing.getring() or y.ring == PolynomialRing.getring() then 
            tex.print('poly')
        end
    end
}
\[ \text{b} = \print{b} \qquad \text{y} =\print{y} \] 
\luaexec{
   if b:getring() == y:getring() then 
    tex.print('success')
   else 
    tex.print('fail') 
   end 
   tex.print('\\[',b.degree:tolatex(),'\\]')
   tex.print('\\[',y.degree:tolatex(),'\\]')
}
b pseudodivide y is causing trouble
\luaexec{
    p = b:zero()
    s = b
    m = s.degree
    n = y.degree
    delta = Integer.max(m-n+Integer.one(), Integer.zero())
    lcb = y:lc()
    sigma = Integer.zero()
}
\[ \print{m} > \print{n} \] 
\luaexec{
    while m >= n and s~= Integer.zero() do 
    local lcs = s:lc()
        p = p * lcb + b:one():multiplyDegree((m-n):asnumber()) * lcs
        s = s * lcb - y * b:one():multiplyDegree((m-n):asnumber()) * lcs
        sigma = sigma + Integer.one()
        m = s.degree
    end
    --qq = lcb^(delta - sigma)*p 
    --rr = lcb^(delta - sigma)*s
    qq = lcb^(delta - sigma)
    rr = lcb^(delta - sigma)
}
\[ \text{lcb} = \print{lcb} \qquad \print{p} \qquad \print{s} \qquad \print{delta -sigma} \]
\[ \text{qq} = \print{qq} \qquad \text{rr} = \print{rr} \] 
\luaexec{
    qq = lcb ^ Integer(2)
}
\[ \print{qq} \] 
\luaexec{
    c = b:pseudodivide(y)
}
\[ \print{c} \] 


\end{document}

\newpage

\begin{CAS}
    vars('A','B','C','D')
    g = A/(x+1) + B/(x+1)^2 + (C*x+D)/(x^2+2*x+2)
    g = g:combine():autosimplify()
\end{CAS}
\[ \print{g}\] 
\[ \print{g.expressions[3]:expand()} \] 

\newpage

\begin{CAS}
    den = (x+1)^2*(x^2+2*x+5)
    num = x^3+2*x^2+x/2+1
    den = den:expand():autosimplify():topolynomial()
    num = num:autosimplify():topolynomial()
    f = PolynomialRing.partialfractions(num,den):autosimplify()
    g = int(f,x)
\end{CAS}
\[ \print{g} = \print*{g} \] 

\begin{CAS}
    vars('x')
    f = (x^2+x+2)/(x^2+2*x+5)
    g = int(f,x):autosimplify()
\end{CAS}
\[ \print{g} \] 
\begin{CAS}
    vars('x')
    num = Poly({2,1,1})
    den = Poly({5,2,1})
    q,r = num:divremainder(den)
    f = q + PolynomialRing.partialfractions(r,den)
    g = int(f:autosimplify(),x)
\end{CAS}
\[ \print{g} = \print*{g} \] 

\begin{CAS}
    vars('a','b','c','d')
    f = -(-a-b)/(c+d)
\end{CAS}
\[ \print{f} = \print*{f} \] 
\parseforest{f}
\bracketset{action character = @}
\begin{forest}
    @\forestresult
\end{forest}

\end{document}
