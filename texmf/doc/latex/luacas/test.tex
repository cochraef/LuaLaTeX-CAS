\documentclass{article}
\usepackage{fullpage}
\usepackage{luacas}
\usepackage{forest}

\begin{document}
Hi
\begin{CAS}
    vars('x')
    f = 1/(x^4+2*x^2+1)
    f = f:autosimplify()
\end{CAS}
\[ \print{f} \] 
\parseforest{f}
\bracketset{action character = @}
\begin{forest}
    @\forestresult
\end{forest}

\luaexec{
    den,dencheck = f.expressions[1]:topolynomial()
    if dencheck then 
        tex.print('pass')
    end
    num = PolynomialRing({Integer.one()}, den.symbol)
}
\[ \print{num/den} \] 
\luaexec{
    gcd = PolynomialRing.gcd(num,den)
    if gcd ~= Integer.one() then 
        f,g = f // gcd, g // gcd 
    end 
    if gcd == Integer.one() then 
        tex.print('pass')
    end
}
\luaexec{
    q,r = num:divremainder(den)
    U = IntegralExpression.integrate(q,x)
    sfden = den:squarefreefactorization()
    pfrac = PolynomialRing.partialfractions(r,den,sfden)
}
\[ \text{pfrac} = \print{pfrac} \] 
\luaexec{
    V = Integer.zero()
    for _, term in ipairs(pfrac.expressions) do
        local i = \#term.expressions
        if i > 1 then
            for j = 1, i-1 do
                local n = term.expressions[j].expressions[1]
                local d = term.expressions[j].expressions[2].expressions[1]
                local p = term.expressions[j].expressions[2].expressions[2]
                local _, s, t = PolynomialRing.extendedgcd(d, d:derivative())
                s = s * n
                t = t * n
                V = V - t / ((p-Integer.one()) * BinaryOperation.POWEXP({d, p-Integer.one()}))
                term.expressions[j+1].expressions[1] = term.expressions[j+1].expressions[1] + s + t:derivative() / (p-Integer.one())
            end
        end
    end
}
\[ \text{V} = \print{V} \] 
\luaexec{
    local W = Integer.zero()
    for _, term in ipairs(pfrac.expressions) do
        a = term.expressions[\#term.expressions].expressions[1]
        tex.print('\\[', a:tolatex(),'\\]')
        b = term.expressions[1].expressions[2].expressions[1]
        tex.print('\\[', b:tolatex(),'\\]')
        y = a - b:derivative() * PolynomialRing({Integer.zero(), Integer.one()}, "z")
        tex.print('\\[',y:tolatex(),'\\]')
        if b.ring == PolynomialRing.getring() or y.ring == PolynomialRing.getring() then 
            tex.print('poly')
        end
    end
}
\[ \text{b} = \print{b} \qquad \text{y} =\print{y} \] 
\luaexec{
   if b:getring() == y:getring() then 
    tex.print('success')
   else 
    tex.print('fail') 
   end 
   tex.print('\\[',b.degree:tolatex(),'\\]')
   tex.print('\\[',y.degree:tolatex(),'\\]')
}
b pseudodivide y is causing trouble
\luaexec{
    p = b:zero()
    s = b
    m = s.degree
    n = y.degree
    delta = Integer.max(m-n+Integer.one(), Integer.zero())
    lcb = y:lc()
    sigma = Integer.zero()
}
\[ \print{m} > \print{n} \] 
\luaexec{
    while m >= n and s~= Integer.zero() do 
    local lcs = s:lc()
        p = p * lcb + b:one():multiplyDegree((m-n):asnumber()) * lcs
        s = s * lcb - y * b:one():multiplyDegree((m-n):asnumber()) * lcs
        sigma = sigma + Integer.one()
        m = s.degree
    end
    --qq = lcb^(delta - sigma)*p 
    --rr = lcb^(delta - sigma)*s
    qq = lcb^(delta - sigma)
    rr = lcb^(delta - sigma)
}
\[ \text{lcb} = \print{lcb} \qquad \print{p} \qquad \print{s} \qquad \print{delta -sigma} \]
\[ \text{qq} = \print{qq} \qquad \text{rr} = \print{rr} \] 
\luaexec{
    qq = lcb ^ Integer(2)
}
\[ \print{qq} \] 
\luaexec{
    c = b:pseudodivide(y)
}
\[ \print{c} \] 
\begin{CAS}
    den = x^4 + 2*x^3 + 2*x^2 + 2*x + 1
    den = den:autosimplify()
    f = 1/den
    f = f:autosimplify()
    g = int(f,x)
\end{CAS}
\[ \print{g} = \print*{g} \] 

\end{document}

\begin{CAS}
    vars('x')
    f = x*(1+2*x^2)^(1/3)
    l = IntegralExpression.trialsubstitutions(f)
\end{CAS}
\[ \lprint{l} \] 
\[ \print{int(f,x)} = \print*{int(f,x)}.\] 

\hrulefill

\begin{CAS}
    vars('x')
    num = Poly({1})
    den = Poly({1,0,0,1})
    g = PolynomialRing.partialfractions(num,den)
    g = g:autosimplify()
    num = Poly({-2,3},x)
    den = Poly({-1,0,0,0,1},x)
    h = PolynomialRing.partialfractions(num,den)
    h = h:autosimplify()
\end{CAS}
%\[ \print{int(h,x)} = \print*{int(h,x)} \] 


\end{document}

\begin{CAS}
    vars('x')
    f = e^x/(e^x*e^x+1)
    l = IntegralExpression.trialsubstitutions(f:autosimplify())
    t = IntegralExpression.substitutionmethod(f,x)
\end{CAS}

\[ \{ \lprint{l} \} \] 
\[ \print{t} \] 

\begin{CAS}
    vars('a','b','c','d')
    x = a^(b*c+b*d)
    x = factor(x)
\end{CAS}
\[ \print{x} = \print*{x} \] 

\begin{CAS}
    vars('x')
    f = cos(x)/(1+sin(x))
    f = f:autosimplify()
    l = IntegralExpression.trialsubstitutions(f)
\end{CAS}
\[ \print{int(f,x)} = \print*{int(f,x)} \] 
\[ \{ \lprint{l} \} \] 

\begin{CAS}
    vars('x')
    f = 1/(1+e^x)
    f = f:autosimplify()
    l = IntegralExpression.trialsubstitutions(f)
    g = f / e^x
    g = g:autosimplify()
\end{CAS}
\[ \print{int(f,x)} = \print*{int(f,x)}\]
\[ \{ \lprint{l} \} \] 
\[ \print{g} \] 

\hrulefill