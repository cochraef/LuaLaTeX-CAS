\documentclass{article}

\usepackage{luacas}
\usepackage{amsmath}
\usepackage{amssymb}

\usepackage[margin=1in]{geometry}
\usepackage[shortlabels]{enumitem}

\usepackage{pgfplots}
\pgfplotsset{compat=1.18}
\usetikzlibrary{positioning,calc}
\usepackage{forest}
\usepackage{minted}
\usemintedstyle{pastie}
\usepackage[hidelinks]{hyperref}
\usepackage{parskip}
\usepackage{multicol}
\usepackage[most]{tcolorbox}
    \tcbuselibrary{xparse}
\usepackage{microtype}

\definecolor{rose}{RGB}{128,0,0}
\definecolor{roseyellow}{RGB}{222,205,99}
\definecolor{roseblue}{RGB}{167,188,214}
\definecolor{rosenavy}{RGB}{79,117,139}
\definecolor{roseorange}{RGB}{232,119,34}
\definecolor{rosegreen}{RGB}{61,68,30}
\definecolor{rosewhite}{RGB}{223,209,167}
\definecolor{rosebrown}{RGB}{108,87,27}
\definecolor{rosegray}{RGB}{84,88,90}

\usepackage[
backend=biber,
style=numeric,
]{biblatex}
\addbibresource{sources.bib}

\newtcolorbox{codebox}[1][sidebyside]{
    enhanced,skin=bicolor,
    #1,
    arc=1pt,
    colframe=brown,
    colback=brown!15,colbacklower=white,
    boxrule=1pt,
    notitle
}

\begin{document}

\subsection{Tutorial 2: Finding Maxima/Minima}

Bob is teaching calculus too, and he wants to give his students many examples of \emph{finding the local max/min of a given function}. But, like Alice, Bob doesn't want to work out a bunch of examples by-hand. Bob decides to try his hand with \texttt{luacas} after having been taught the basics by Alice. 

Bob decides to stick with polynomials for these examples; if anything because those functions are in the wheel-house of \texttt{luacas}. In particular, Bob decides that the \emph{derivative} of the function he wants to use should be a composition of quadratics. This ought to ensure that the roots of that derivative are expressible in a nice way. 

Accordingly, Bob declares variables and chooses two quadratic polynomials to compose, say $f$ and $g$, and sets $dh = g \circ f$:

\begin{minted}{latex}
\begin{CAS}
    vars('x')
    f = x^2+2*x-2
    g = x^2-1
    subs = {[x] = f}
    dh = substitute(subs,g)
\end{CAS}
\end{minted}
\begin{CAS}
    vars('x')
    f = x^2+2*x-2
    g = x^2-1
    subs = {[x] = f}
    dh = substitute(subs,g)
\end{CAS}

Bob wants to compute $h$, the integral of $dh$. Bob could certainly compute this quantity by-hand, but why hardcode that information into the document when \texttt{luacas} can do this for you? So Bob uses the \texttt{int} command and shifts the result (with some malice aforethought):

\begin{minted}{latex}
\begin{CAS}
    h = int(dh,x) + 10
\end{CAS}
\end{minted}
\begin{CAS}
    h = int(dh,x) + 10
\end{CAS}

Bob is curious to know the value of $h$. So he uses \mintinline{latex}{\print{h}} to produce:
\begin{codebox}
\begin{minted}[fontsize=\small]{latex}
\[ \print{h} \] 
\end{minted}
    \tcblower
\[\print{h} \] 
\end{codebox}
This isn't exactly what Bob had in mind. It occurs to Bob that he may need to simplify the expression $h$, so he tries:

\begin{codebox}
\begin{minted}[fontsize=\small]{latex}
\begin{CAS}
    h = simplify(int(dh,x)+10)
\end{CAS}
\[ \print{h} \] 
\end{minted}
    \tcblower
\begin{CAS}
    h = simplify(h)
\end{CAS}
\[\print{h} \] 
\end{codebox}

That's more like it! Now, Bob wants to find the roots to $dh$. Bob uses the \texttt{roots} command to do this:

\begin{minted}[fontsize=\small]{latex}
\begin{CAS}
    r = roots(dh)
\end{CAS}
\end{minted}
\begin{CAS}
    r = roots(dh)
\end{CAS}

But then Bob wonders to himself, ``How do I actually retrieve the roots of $dh$ from \texttt{luacas}?'' The assignment \mintinline{lua}{r = roots(dh)} stores the roots of the polynomial $dh$ in a table named \texttt{r}:

\begin{codebox}[]
\begin{minted}[fontsize=\small]{latex}
\[ \print{r[1]}, \quad \print{r[2]}, \quad \print{r[3]}, \quad\print{r[4]} \] 
\end{minted}
    \tcblower 
    \begin{CAS}
        r = roots(dh)
    \end{CAS}
    \[ \print{r[1]}, \quad \print{r[2]}, \quad \print{r[3]}, \quad \print{r[4]} \] 
\end{codebox}

Splendid! Bob would now like to evaluate the function $h$ at these roots (for these are the local max/min values of $h$). Here's Bob's first thought:

\begin{codebox}
\begin{minted}[fontsize=\small]{latex}
\begin{CAS}
  v = simplify(substitute({[x]=r[1]},h))
\end{CAS}
\[ \print{v} \] 
\end{minted}
    \tcblower 
    \begin{CAS}
        v = simplify(substitute({[x]=r[1]},h))
    \end{CAS}
    \[ \print{v} \] 
\end{codebox}

What the heck?! Bob is (understandably) confused. But here's where Bob learns a valuable lesson\dots 

\subsubsection{A brief interlude: Lua numbers vs \texttt{luacas Integers}}

The \LaTeX{} environment \mintinline{latex}{\begin{CAS}..\end{CAS}} is really a glorified Lua environment. The ``glory'' comes in how the contents of the environment are parsed in a special manner to make interacting with the CAS (mostly) easy. Bob has encountered a situation where that interaction is not as easy as we'd like. 

For comparison, consider the following:
\begin{multicols}{2}
Here's some code using the \mintinline{latex}{\begin{CAS}..\end{CAS}}:
\begin{codebox}\small
    \begin{minted}[fontsize=\small]{latex}
\begin{CAS}
    vars('y')
    a = 1
    b = y+a
\end{CAS}
\[ \print{b} \] 
\end{minted}
    \tcblower 
    \begin{CAS}
        vars('y')
        a = 1
        b = y+a
    \end{CAS}
    \[ \print{b} \] 
\end{codebox}
Here's that same code but using \mintinline{latex}{\directlua} instead:
\begin{codebox}\small
\begin{minted}[fontsize=\small]{latex}
\directlua{
    vars('y')
    a = Integer(1)
    b = y+a
}
\[ \print{b} \] 
\end{minted}
    \tcblower 
    \directlua{
    vars('y')
    a = Integer(1)
    b = y+a
}
\[ \print{b} \] 
\end{codebox}
\end{multicols}
The essential difference being:
\begin{itemize}
    \item Using \mintinline{latex}{\begin{CAS}..\end{CAS}}, the parser automatically interprets any digit strings as an \texttt{Integer}; this is a special class defined within the bowels of \texttt{luacas}. Ultimately, it allows for us to define things like the addition of an \texttt{Integer} and an \texttt{Expression} (in this case, the result is a new \texttt{Expression}) as well as arbitrary precision arithmetic.
    \item Using \mintinline{latex}{\directlua}, there is no parsing, so the user (aka Bob) is responsible for telling \texttt{luacas} what to interpret as an \texttt{Integer} versus what to interpret as a normal Lua \texttt{number}. 
\end{itemize}
Generally speaking, we like what the parser in \mintinline{latex}{\begin{CAS}..\end{CAS}} does: it keeps us from having to wrap all integers in \texttt{Integer(..)} (among other things). But the price we pay is that the parser indiscriminately wraps \emph{all} (or rather, most) digit strings in \texttt{Integer(..)}. This causes a problem in the following line in Bob's code:
\begin{minted}{lua}
v = simplify(substitute({[x]=r[1]},h))
\end{minted} 
The parser sees \mintinline{lua}{r[1]} and interprets \texttt{1} as \texttt{Integer(1)} -- but \mintinline{lua}{r[Integer(1)]} is \texttt{nil}, so no substitution is performed. 

The good news is that, excluding the annoyance between \texttt{Integer} and Lua number, interacting with the CAS via \mintinline{latex}{\directlua} is not much different than interacting with it via \mintinline{latex}{\begin{CAS}..\end{CAS}}

\subsubsection*{Back to the tutorial...}

After that enlightening interlude, Bob realizes that he should probably use \mintinline{latex}{\directlua} (or \mintinline{latex}{\luaexec} from the \texttt{luacode} package) to build a table of values:

\begin{codebox}[]
\begin{minted}[fontsize=\small]{latex}
\directlua{
    v = {}
    for i=1,4 do 
        table.insert(v,simplify(substitute({[x]=r[i]},h)))
    end
}
\[ \print{v[1]}, \quad \print{v[2]}, \quad \print{v[3]}, \quad\print{v[4]} \]   
\end{minted}
    \tcblower 
    \directlua{
    v = {}
    for i=1,4 do 
        table.insert(v,simplify(substitute({[x]=r[i]},h)))
    end
}
\[ \print{v[1]}, \quad \print{v[2]}, \quad \print{v[3]}, \quad \print{v[4]} \] 
\end{codebox}
Great! But still; Bob doesn't want to just pretty-print the roots of $dh$. Bob is determined to plot the results -- he wants to hammer home the point that the roots of $dh$ point to the local extrema of $h$. 

Luckily, Bob is familiar with some of the fantastic graphics tools in the \LaTeX{} ecosystem, like \texttt{pgfplots} and \texttt{asymptote}. But then Bob begins to wonder, ``How can I yoink results out of \texttt{luacas} so that I may yeet them into something like \texttt{pgfplots}?''
Bob is delighted to find the following commands: \mintinline{latex}{\fetch} and \mintinline{latex}{\store}.

Whereas the \mintinline{latex}{\print} command relies on the \texttt{luacas} method \mintinline{lua}{tolatex()}, the commands \mintinline{latex}{\fetch} and \mintinline{latex}{\store} rely on the \texttt{luacas} function \mintinline{lua}{tostring()}. Bob can view the output of \mintinline{lua}{tostring()} using the \mintinline{latex}{\vprint} command ({\bf v}erbatim {\bf print}). For example, \mintinline{latex}{\vprint{h}} produces:
\vprint{h} 
This is more-or-less what Bob wants -- but he doesn't want the verbatim output printed to his document, Bob just wants the contents of \mintinline{lua}{tostring(h)}. Here's where \mintinline{latex}{\fetch} comes in. The command \mintinline{latex}{\fetch{h}} is equivalent to:
\begin{minted}{latex}
    \directlua{
        tex.print(tostring(h))
    }
\end{minted}
For comparison, the command \mintinline{latex}{\print{h}} is equivalent to:
\begin{minted}{latex}
    \directlua{
        tex.print(h:tolatex())
    }
\end{minted}
For Bob's purposes, \mintinline{latex}{\fetch{h}} is exactly what he needs:

\begin{codebox}\small
\begin{minted}[breaklines,fontsize=\small]{latex}
\begin{tikzpicture}[scale=0.9]
  \begin{axis}[legend pos = north west]
    \addplot [domain=-3.5:1.5,samples=100]
      {\fetch{h}};
    \addlegendentry{$f$};
    \addplot[densely dashed] 
      [domain=-3.25:1.25,samples=100]
      {\fetch{dh}};
    \addlegendentry{$df/dx$};
    \addplot[gray,dashed,thick]
      [domain=-3.5:1.5] {0};
  \end{axis}
\end{tikzpicture}
\end{minted}
\tcblower
\begin{tikzpicture}[scale=0.9]
    \begin{axis}[legend pos = north west]
        \addplot [domain=-3.5:1.5,samples=100] {\fetch{h}};
        \addlegendentry{$f$};
        \addplot[densely dashed] 
            [domain=-3.25:1.25,samples=100] {\fetch{dh}};
        \addlegendentry{$df/dx$};
        \addplot[gray,dashed,thick]
            [domain=-3.5:1.5] {0};
    \end{axis}
\end{tikzpicture}
\end{codebox}
Alternatively, Bob could use \mintinline{latex}{\store}. The \mintinline{latex}{\store} command will \emph{fetch} the contents of its mandatory argument and store it in a macro of the same name. 
\begin{minted}[fontsize=\small]{latex}
\store{h}
\store{dh}
\end{minted}
Now the macros \mintinline{latex}{\h} and \mintinline{latex}{\dh} can be used in place of \mintinline{latex}{\fetch{h}} and \mintinline{latex}{\fetch{dh}}, respectively. An optional argument can be used to store contents in a macro under a different name. This is useful for situations like the following:
\begin{minted}[fontsize=\small]{latex}
\store{r[1]}[rootone]
\end{minted}
Now \mintinline{latex}{\rootone} can be used in place of \mintinline{latex}{\fetch{r[1]}}. But Bob wants to fetch all the values stored in \texttt{r} (and \texttt{v}, for that matter). In this case, Bob can use:
\begin{minted}{latex}
\store{r}
\store{v}
\end{minted}
The command \mintinline{latex}{\store{r}} is equivalent to:
\begin{minted}[fontsize=\small]{latex}
\def\r{{ \fetch{r[1]}, \fetch{r[2]}, \fetch{r[3]}, \fetch{r[4]} }}
\end{minted}
The contents of the \LaTeX{} macro \mintinline{latex}{\r} can be accessed with \mintinline{latex}{\pgfmathsetmacro}. For example:

\begin{codebox}
\begin{minted}[fontsize=\small]{latex}
\begin{tikzpicture}[scale=0.6]
  \draw [dashed,latex-latex]
    (-7,0) -- (4,0);
  \foreach \k in {0,1,2,3}{
    \pgfmathsetmacro\a{\r[\k]}
    \draw (\a,0) circle (\a);
  }
  \foreach \x in {-6,...,3}{
    \draw[fill,orange]
      (\x,0) circle (2pt)
      node[below] {\footnotesize$\x$};
  }
\end{tikzpicture}
\end{minted}
    \tcblower
    \store{r}
    \begin{center}
    \begin{tikzpicture}[scale=0.65]
        \draw [dashed,latex-latex] (-7,0) -- (4,0);
        \foreach \k in {0,1,2,3}{
            \pgfmathsetmacro\a{\r[\k]}
            \draw (\a,0) circle (\a);
        }
        \foreach \x in {-6,...,3}{
            \draw[fill,orange] (\x,0) circle (2pt)
            node[below] {\footnotesize$\x$};
        }
    \end{tikzpicture}
\end{center}
\end{codebox}

Alternatively, one could appeal directly to the \mintinline{lua}{tostring()} function in \texttt{luacas} and iterate over tables like \texttt{r} using Lua itself. This can often be a simpler solution, and it is exactly what Bob does in his complete project shared below:
\begin{codebox}[frame hidden,breakable]
\begin{minted}[breaklines,fontsize=\small]{latex}
Consider the function $f(x)$ defined by:
\begin{CAS}
  vars('x')
  f = x^2+2*x-2
  g = x^2-1
  subs = {[x] = f}
  dh = expand(substitute(subs,g))
  h = simplify(int(dh,x)+10)
\end{CAS}
$\displaystyle f(x) = \print{h}$.
\begin{multicols}{2}
  Note that: 
  \[ f'(x) = \print{dh}.\] 
  The roots to $f'(x)=0$ equation are:
  \begin{CAS}
      r = roots(dh)
  \end{CAS}
  \[ \print{r[1]}, \quad \print{r[2]}, \quad \print{r[3]}, \quad  \print{r[4]}.\] 
  Recall: $f'(x_0)$ measures the slope of the tangent line to $y=  (x)$ at $x=x_0$. The values $r$ where $f'(r)=0$ correspond to  places where the slope of the tangent line to $y=f(x)$ is horizontal (see the illustration). This gives us a method for identifying locations where the graph $y=f(x)$ attains a peak  (local maximum) or a valley (local minimum). 
  \directlua{
      v = {}
      for i=1,4 do 
          table.insert(v,simplify(substitute({[x]=r[i]},h)))
      end
  }
  \columnbreak 
  \store{h}\store{dh}
  \begin{tikzpicture}[scale=0.95]
    \begin{axis}[legend pos = north west]
      \addplot 
        [domain=-3.5:1.5,samples=100] {\h};
      \addlegendentry{$f$};
      \addplot[densely dashed] 
        [domain=-3.25:1.25,samples=100] {\dh};
      \addlegendentry{$df/dx$};
      \addplot[gray,dashed,thick]
        [domain=-3.5:1.5] {0};
      \luaexec{for i=1,4 do 
        tex.print("\\draw[fill=purple,purple]",
          "(axis cs:{",tostring(r[i]),"},0) circle (1.5pt)",
          "(axis cs:{",tostring(r[i]),"},{",tostring(v[i]),"}) circle (1.5pt)",
          "(axis cs:{",tostring(r[i]),"},{",tostring(v[i]),"}) edge[dashed] (axis cs:{",tostring(r[i]),"},0);")
        end}
    \end{axis}
  \end{tikzpicture}
\end{multicols}
\end{minted}
\end{codebox}
And here is Bob's completed project:
\begin{tcolorbox}[colback=rosenavy!10,
    colframe=rosenavy,
    arc=1pt,
    frame hidden]
    {\bf Tutorial 2:} {\itshape A local max/min diagram for Bob}.
    \vskip 0.2cm
    Consider the function $f(x)$ defined by:
    \begin{CAS}
        vars('x')
        f = x^2+2*x-2
        g = x^2-1
        subs = {[x] = f}
        dh = expand(substitute(subs,g))
        h = simplify(int(dh,x)+10)
    \end{CAS}
    $\displaystyle f(x) = \print{h}$.
\begin{multicols}{2}
    Note that: 
    \[ f'(x) = \print{dh}.\] 
    The roots to $f'(x)=0$ equation are:
    \begin{CAS}
        r = roots(dh)
    \end{CAS}
    \[ \print{r[1]}, \quad \print{r[2]}, \quad \print{r[3]}, \quad \print{r[4]}.\] 
    Recall: $f'(x_0)$ measures the slope of the tangent line to $y=f(x)$ at $x=x_0$. The values $r$ where $f'(r)=0$ correspond to places where the slope of the tangent line to $y=f(x)$ is horizontal (see the illustration). This gives us a method for identifying locations where the graph $y=f(x)$ attains a peak (local maximum) or a valley (local minimum). 
    \directlua{
    v = {}
    for i=1,4 do 
        table.insert(v,simplify(substitute({[x]=r[i]},h)))
    end
    }
    \columnbreak 
    \store{h}\store{dh}
    \begin{tikzpicture}[scale=0.95]
        \begin{axis}[legend pos = north west]
            \addplot 
            [domain=-3.5:1.5,samples=100] {\h};
            \addlegendentry{$f$};
            \addplot[densely dashed] 
            [domain=-3.25:1.25,samples=100] {\dh};
            \addlegendentry{$df/dx$};
            \addplot[gray,dashed,thick]
            [domain=-3.5:1.5] {0};
            \luaexec{for i=1,4 do 
                tex.print("\\draw[fill=purple,purple]",
                    "(axis cs:{", tostring(r[i]) ,"},0) circle (1.5pt)",
                    "(axis cs:{", tostring(r[i]) ,"},{", tostring(v[i]), "}) circle (1.5pt)",
                    "(axis cs:{", tostring(r[i]) ,"},{", tostring(v[i]), "}) edge[dashed] (axis cs:{", tostring(r[i]) ,"},0);")
                end}
        \end{axis}
    \end{tikzpicture}
\end{multicols}
\end{tcolorbox}

\end{document}