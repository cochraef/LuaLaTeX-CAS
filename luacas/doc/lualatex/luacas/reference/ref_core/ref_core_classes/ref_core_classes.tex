\documentclass{article}

\usepackage{luacas}
\usepackage{amsmath}
\usepackage{amssymb}

\usepackage[margin=1in]{geometry}
\usepackage[shortlabels]{enumitem}

\usepackage{pgfplots}
\pgfplotsset{compat=1.18}
\usetikzlibrary{positioning,calc}
\usepackage{forest}
\usepackage{minted}
\usemintedstyle{pastie}
\usepackage[hidelinks]{hyperref}
\usepackage{parskip}
\usepackage{multicol}
\usepackage[most]{tcolorbox}
    \tcbuselibrary{xparse,documentation}
\usepackage{microtype}
\usepackage{makeidx}
\usepackage{fontawesome5}
\usepackage{marginnote}

\usepackage[
backend=biber,
style=numeric,
]{biblatex}
\addbibresource{sources.bib}

\definecolor{rose}{RGB}{128,0,0}
\definecolor{roseyellow}{RGB}{222,205,99}
\definecolor{roseblue}{RGB}{167,188,214}
\definecolor{rosenavy}{RGB}{79,117,139}
\definecolor{roseorange}{RGB}{232,119,34}
\definecolor{rosegreen}{RGB}{61,68,30}
\definecolor{rosewhite}{RGB}{223,209,167}
\definecolor{rosebrown}{RGB}{108,87,27}
\definecolor{rosegray}{RGB}{84,88,90}

\definecolor{codegreen}{HTML}{49BE25}

\newtcolorbox{codebox}[1][sidebyside]{
    enhanced,skin=bicolor,
    #1,
    arc=1pt,
    colframe=brown,
    colback=brown!15,colbacklower=white,
    boxrule=1pt,
    notitle
}

\newtcolorbox{codehead}[1][]{
    enhanced,
    frame hidden,
    colback=rosegray!15,
    boxrule=0mm,
    leftrule=5mm,
    rightrule=5mm,
    boxsep=0mm,
    arc=0mm,
    outer arc=0mm,
    left=3mm,
    right=3mm,
    top=1mm,
    bottom=1mm,
    toptitle=1mm,
    bottomtitle=1mm,
    oversize,
    #1
}

\usepackage{varwidth}

\newtcolorbox{newcodehead}[2][]{
    enhanced,
    frame hidden,
    colback=rosegray!15,
    boxrule=0mm,
    leftrule=5mm,
    rightrule=5mm,
    boxsep=0mm,
    arc=0mm,
    outer arc=0mm,
    left=3mm,
    right=3mm,
    top=1mm,
    bottom=1mm,
    toptitle=1mm,
    bottomtitle=1mm,
    oversize,
    #1,
    fonttitle=\bfseries\ttfamily\footnotesize,
    coltitle=rosegray,
    attach boxed title to top text right,
    boxed title style={frame hidden,size=small,bottom=-1mm,
    interior style={fill=none,
    top color=white,
    bottom color=white}},
    title={#2}
}

\makeindex

\newcommand{\coderef}[2]{%
\begin{codehead}[sidebyside,segmentation hidden]%
    \mintinline{lua}{#1}%
    \tcblower%
    \begin{flushright}%
    \mintinline{lua}{#2}%
    \end{flushright}%
\end{codehead}%
}

\newcommand{\newcoderef}[3]{%
\begin{newcodehead}[sidebyside,segmentation hidden]{#3}%
    \mintinline{lua}{#1}%
    \tcblower%
    \begin{flushright}%
    \mintinline{lua}{#2}%
    \end{flushright}%
\end{newcodehead}%
}

\begin{document}

\subsection{Core Classes}

There are several classes in the core module; but only some classes are concrete:

\begin{multicols}{2}
    \begin{center}
        \underline{Abstract classes:}
    \begin{itemize}
        \item \texttt{Expression}
        \item \texttt{AtomicExpression}
        \item \texttt{CompoundExpression}
        \item \texttt{ConstantExpression}
    \end{itemize}

        \underline{Concrete classes:}
    \begin{itemize}
        \item \texttt{SymbolExpression}
        \item \texttt{BinaryOperation}
        \item \texttt{FunctionExpression}
    \end{itemize}
\end{center}
\end{multicols}

The abstract classes provide a unified interface for the concrete classes (expressions) using inheritance. \emph{Every} expression in \texttt{luacas} inherits from either {\ttfamily AtomicExpression} or {\ttfamily CompoundExpression} which, in turn, inherit from {\ttfamily Expression}. 

\coderef{function SymbolExpression:new(string)}{return SymbolExpression}
\index{Core!Classes!\texttt{SymbolExpression}}
\addcontentsline{toc}{subsubsection}{\ttfamily SymbolExpression}

Creates a new \texttt{SymbolExpression}. For example:
\begin{codebox}[]
\begin{minted}[breaklines,fontsize=\small]{lua}
foo = SymbolExpression("bar")
tex.sprint("The Lua variable ``foo'' is the SymbolExpression: ", foo:tolatex(),".")
\end{minted}
\tcblower
\directlua{
foo = SymbolExpression("bar")
tex.sprint("The Lua variable 'foo' is the SymbolExpression: ", foo:tolatex(),".")
}
\end{codebox}

\subsubsection*{Fields}

\texttt{SymbolExpression}s have only one field: \texttt{symbol}. In the example above, the string \mintinline{lua}{"bar"} is stored in \mintinline{lua}{foo.symbol}. 

\subsubsection*{Parsing}

The command \mintinline{lua}{vars()} in \texttt{test.parser} creates a new \texttt{SymbolExpression} for every string in the argument; each such \texttt{SymbolExpression} is assigned to a variable of the same name. For example:

\begin{minted}{lua}
vars('x','y')
\end{minted}

is equivalent to:

\begin{minted}{lua}
x = SymbolExpression("x")
y = SymbolExpression("y")
\end{minted}

\newcoderef{function BinaryOperation:new(operation, expressions)}{return BinaryOperation}{operation function, expressions table<number,Expression>}
\index{Core!Classes!\texttt{BinaryOperation}}
\addcontentsline{toc}{subsubsection}{\ttfamily BinaryOperation}

Creates a new \texttt{BinaryOperation} expression. For example:

\begin{codebox}
\begin{minted}[fontsize=\small]{lua}
vars('x','y','z')
w = BinaryOperation(
    BinaryOperation.ADD,
    {BinaryOperation(
        BinaryOperation.MUL,
        {x,y}
    ),y,z}
)
tex.print("\\[w=",w:tolatex(),"\\]")
\end{minted}
\tcblower
\directlua{
vars('x','y','z')
w = BinaryOperation(
    BinaryOperation.ADD,
    {BinaryOperation(
        BinaryOperation.MUL,
        {x,y}
    ),y,z}
)
tex.print("\\[w=",w:tolatex(),"\\]")
}
\end{codebox}
The variable \texttt{operation} must be a function \mintinline{lua}{function f(a,b)} assigned to one of the following types:
\bgroup
\setdescription{style=multiline,
        topsep=10pt,
        leftmargin=4.5cm,
        font=\ttfamily
        }
\begin{description}
    \item[BinaryOperation.ADD:] \mintinline{lua}{return a + b}
    \item[BinaryOperation.SUB:] \mintinline{lua}{return a - b}
    \item[BinaryOperation.MUL:] \mintinline{lua}{return a * b}
    \item[BinaryOperation.DIV:] \mintinline{lua}{return a / b}
    \item[BinaryOperation.IDIV:] \mintinline{lua}{return a // b}
    \item[BinaryOperation.MOD:] \mintinline{lua}{return a % b}
    \item[BinaryOperation.POW:] \mintinline{lua}{return a ^ b}
\end{description}
\egroup
The variable \texttt{expressions} must be a table of \texttt{Expression}s index by Lua numbers.

\subsubsection*{Fields}

\texttt{BinaryOperation}s have the following fields: \texttt{name}, \texttt{operation}, and \texttt{expressions}. In the example above, we have:
\begin{itemize}
    \item the variable \texttt{expressions} is stored in \mintinline{lua}{w.expressions};
    \item \mintinline{lua}{w.name} stores the string \mintinline{lua}{"+"}; and 
    \item \mintinline{lua}{w.operation} stores the function:
    \begin{minted}{lua}
BinaryOperation.ADD = function(a, b)
    return a + b
end
    \end{minted}
\end{itemize}

\begin{multicols}{2}
The entries of \texttt{w.expressions} can be used/fetched in a reasonable way:
\begin{codebox}[]
    \begin{minted}[fontsize=\small]{latex}
$\print{w.expressions[1]} \quad
 \print{w.expressions[2]} \quad
 \print{w.expressions[3]}$
    \end{minted}
    \tcblower
$\print{w.expressions[1]} \quad
 \print{w.expressions[2]} \quad
 \print{w.expressions[3]}$
\end{codebox}

\begin{center}
    \bracketset{action character = @}
    \parseshrub{w}
    \begin{forest}
        for tree = {font = \ttfamily,
            draw,
            rounded corners = 1pt,
            fill = gray!20,
            l sep = 1.5cm,
            s sep = 2cm}
        @\shrubresult
    \end{forest}
\end{center}
\end{multicols}

\subsubsection*{Parsing}

Thank goodness for this. Creating new \texttt{BinaryOperation}s isn't nearly as cumbersome as the above would indicate. Using Lua's powerful metamethods, we can parse expressions easily. For example, the construction of \texttt{w} given above can be done much more naturally using:
\begin{codebox}
\begin{minted}[fontsize=\small]{lua}
vars('x','y','z')
w = x*y+y+z
tex.print("\\[w=", w:tolatex(), "\\]")
\end{minted}
\tcblower
\directlua{
    vars('x','y','z')
    w = x*y+y+z
    tex.print("\\[w=", w:tolatex(), "\\]")
}
\end{codebox}
\reversemarginpar
{\bf Warning:}\marginnote{\color{rose}\faExclamationTriangle} There are escape issues to be aware of with the operator \mintinline{latex} with reckless abandon. But when using the operator \mintinline{latex} in place of \mintinline{latex}{%}:

\begin{codebox}
\begin{minted}[breaklines,fontsize=\small]{latex}
\begin{CAS}
    a = 17
    b = 5
    c = a \% b
\end{CAS}
\[ \print{c} \equiv \print{a} \bmod{\print{b}} \]
\end{minted}
\tcblower
\begin{CAS}
    a = 17
    b = 5
    c = a \% b
\end{CAS}
\[ \print{c} \equiv 
    \print{a} \bmod{\print{b}} \] 
\end{codebox}
The above escape will {\bf not} work with \mintinline{latex}{\directlua}, but it will work for \mintinline{latex}{\luaexec} from the \texttt{luacode} package. Indeed, the \texttt{luacode} package was designed (in part) to make escapes like this more manageable. Here is the equivalent code using \mintinline{latex}{\luaexec}:
\begin{codebox}[]
\begin{minted}[fontsize=\small]{lua}
a = Integer(17)
b = Integer(5)
c = a \% b
tex.print("\\[",c:tolatex(),"\\equiv",a:tolatex(), "\\bmod{",b:tolatex(),"} \\]")
\end{minted}
\tcblower
\luaexec{
a = Integer(17)
b = Integer(5)
c = a \% b
tex.print("\\[", c:tolatex(), "\\equiv", a:tolatex(), "\\bmod{", b:tolatex(), "} \\]")
}
\end{codebox}

\newcoderef{function FunctionExpression:new(name,expressions)}{return FunctionExpression}{name string|SymbolExpression, expressions table<number,Expression>}
\index{Core!Classes!\texttt{FunctionExpression}}
\addcontentsline{toc}{subsubsection}{\ttfamily FunctionExpression}

Creates a generic function. For example:
\begin{codebox}
    \begin{minted}[fontsize=\small]{lua}
vars('x','y')
f = FunctionExpression('f',{x,y})
tex.print("\\[",f:tolatex(),"\\]")
    \end{minted}
    \tcblower
    \luaexec{
    vars('x','y')
    f = FunctionExpression('f',{x,y})
    tex.print("\\[",f:tolatex(),"\\]")
    }
\end{codebox}
The variable \texttt{name} can be a string (like above), or another \texttt{SymbolExpression}. But in this case, the variable \texttt{name} just takes the value of the string \mintinline{lua}{SymbolExpression.symbol}. The variable \texttt{expressions} must be a table of \texttt{Expression}s indexed by Lua numbers.

\subsubsection*{Fields}
\texttt{FunctionExpression}s have the following fields: \texttt{name}, \texttt{expressions}, \texttt{variables}, \texttt{derivatives}. In the example above, we have:
\begin{itemize}
    \item the variable \texttt{name}, i.e. the string \mintinline{lua}{'f'}, is stored in \mintinline{lua}{f.name}; and
    \item the variable \texttt{expressions}, i.e. the table \mintinline{lua}{{x,y}} is stored in \mintinline{lua}{f.expressions}.
\end{itemize}

Wait a minute, what about \texttt{variables} and \texttt{derivatives}!? The field \texttt{variables} essentially stores a copy of the variable \texttt{expressions} \textit{as long as} the entries in that table are atomic. If they aren't, then \texttt{variables} will default to $x,y,z$, or $x_1,x_2,\ldots$ if the number of variables exceeds $3$. For example:

\begin{codebox}
    \begin{minted}[fontsize=\small]{lua}
vars('s','t')
f = FunctionExpression('f',{s*s,s+t+t})
tex.print("The variables of f are:")
for _,symbol in ipairs(f.variables) do 
    tex.print(symbol:tolatex())
end
    \end{minted}
    \tcblower
\luaexec{
    vars('s','t')
    f = FunctionExpression('f',{s*s,s+t+t})
    tex.print("The variables of f are:")
    for _,symbol in ipairs(f.variables) do 
        tex.print(symbol:tolatex())
    end
}
\end{codebox}
The field \texttt{derivatives} is a table of \texttt{Integer}s indexed by Lua numbers whose length equals \mintinline{lua}{#o.variables}. The default value for this table is a table of (\texttt{Integer}) zeros. So for the example above, we have:
\begin{codebox}
    \begin{minted}[fontsize=\small]{lua}
for _,integer in ipairs(f.derivatives) do 
  if integer == Integer.zero() then
    tex.print("I'm a zero.\\newline")
  end
end
\end{minted}
\tcblower
\luaexec{
    for _,integer in ipairs(f.derivatives) do 
        if integer == Integer.zero() then
            tex.print("I'm a zero.\\newline")
        end
    end
}
\end{codebox}
We can change the values of \texttt{variables} and \texttt{derivatives} manually (or more naturally by other gizmos found in \texttt{luacas}). For example, keeping the variables from above, we have:
\begin{multicols}{2}
\begin{codebox}[]
    \begin{minted}[fontsize=\small]{lua}
f.derivatives = {Integer.one(),
    Integer.one()}
tex.print("\\[",
    f:simplify():tolatex(),
    "\\]")
\end{minted}
\tcblower
\luaexec{
    f.derivatives = {Integer.one(),Integer.one()}
    tex.print("\\[", f:simplify():tolatex(), "\\]")
}
\end{codebox}

\begin{center}
\parseshrub{f}
\bracketset{action character = @}
\begin{forest}
    for tree = {font = \ttfamily,
        draw,
        rounded corners = 1pt,
        fill = gray!20,
        l sep = 1.5cm,
        s sep = 0.75cm}
    @\shrubresult
\end{forest}
\end{center}
\end{multicols}

\subsubsection*{Parsing}

Thank goodness for this too. The parser nested within the \LaTeX{} environment \mintinline{latex}{\begin{CAS}..\end{CAS}} allows for fairly natural function assignment; the name of the function must be declared in \mintinline{lua}{vars(...)} (or rather, as a \texttt{SymbolExpression}) beforehand:
\begin{codebox}
    \begin{minted}[fontsize=\small]{latex}
\begin{CAS}
    vars('s','t','f')
    f = f(s^2,s+2*t)
    f.derivatives = {1,1}
\end{CAS}
\[ \print{f} \] 
\end{minted}
\tcblower
\begin{CAS}
    vars('s','t','f')
    f = f(s^2,s+2*t)
    f.derivatives = {1,1}
\end{CAS}
\[ \print{f} \] 
\end{codebox}

\end{document}